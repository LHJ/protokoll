\documentclass{protokoll}

\usepackage[utf8]{inputenc}

\bild{logo.pdf}     % Ändra om man vill ha en annan bild på framsidan.
\plats{T1}          % Plats som mötet hålls i
\typ{Styrelsemöte}  %Typ av möte: Styrelsemöte|Årsmöte|medlemsmöte etc.
\datum{2018-09-11}  %Datum, form YYYY-MM-DD
\tid{18:01}{18:41}   %Tid: {START}{SLUT}.
\organ{Styrelsen}   %Det organ som har hand om mötet: Styrelsen|Mötet|Medlemmarna
\organisation{Luleå Universitets Datorförening}
\forordf{August Eriksson} 	            % Föreningsordförande
\ordf{August Eriksson}%[titel] 	    % Mötesordförande  | (titel)
\sekr{Jens Lindholm}%[titel] 		% Mötessekreterare | (titel)
\justA{Anton Johansson}{Ledamot} 		% Justerare A      | titel
\justB{Edvin Åkerström}{Ledamot} 		% Justerare B      | titel

% Lista över övriga personer som närvarade vid mötet.
% Om det anses för många vid ett möte skriv: \person{<ANTAL> övriga medlemmar.}{}
\person{Frida Mikkelä}{Vice Ordförande}
\person{Fredrik Pettersson}{Kassör}
%\person{Emil Kitti}{Huvudsystemansvarig}
%\person{Anton Johansson}{Ledamot}
%\person{Edvin Åkerström}{Ledamot}
\person{Erik Viklund}{Medlem}
\person{Oscar Brink}{Medlem}




%\adjung{} % Adjungerad person
%\adjung{} % Nästa adjungerade person
%\behorighet{Trump} % Om mötet är obehörigt, inkludera denna rad med anledning(-arna).

\begin{document}
% Punkter som hanteras automagiskt:
% Mötets högtidliga öppnande - när mötet öppnats
% Formalia
%       Val av mötesordförande - den som valdes tills mötesordförande
%       Val av mötessekreterare - den som valdes till mötessekreterare (alltså du)
%       Val avtvå justeringspersoner tillika rösträknare - de som valdes till protokolljusterare
%       Mötets behöriga utlysande - om mötet är behörigt är denna automagisk annars använd \behorighet
%       (Eventuella adjungeringar) - en samling av de som inadjungerats; kontrolleras via \adjung kommadot
% Mötets högtidliga avslutande - när mötet avslutats
\newpage  


\punkt{Godkännande av dagordning}
% Skriv eventuella ändringar i dagordningen och ändra beslutet.
\begin{beslut}
     \att fastslå dagordningen som skickats ut.
\end{beslut}

\punkt{Bordlagda ärenden}
% Har det bordlagts ärenden från förra mötet?
Inga bordlagda ärenden.

\punkt{Medverkan i NCPC}
Mötet är positivt till att vara med som en hub. Fredrik kommer kontakta tidigare representanter för att höra mer om hur det går till.

\punkt{Rättning av protokoll i GitLab}
Inga protokoll har rättats sedan flytten till Gitlab. Med anledning av detta håller August i en demonstration för mötet hur rättning går till i GitLab. 

\punkt{Pågående projekt}
\subpunkt{LEDs i korridor och LED-skylt}
LUDD:s nya LED-skylt är nu uppsatt. Nuvarande nätaggregat och LED-slinga som används ger inte önskad "fade"-effekt på skylten, varefter en bättre lösning eftertraktas. %Som jag förstod det skulle LED-skylten ha ett API för att medlemmar skulle kunna ändra färg, eller nåt, också.

\subpunkt{Profilering}
Inget att rapportera. 

\subpunkt{Ny statisk hemsida}
Inget att rapportera. 

\subpunkt{Övriga projekt}
\subsubsection{LaTeX-kurs}
Frida nämner att LaTeX-kursen fortfarande inte har något spikat datum trots att första läsperioden redan har börjat. Mötet diskuterar lämpligt datum för kursen och konstaterar att förra årets LaTeX-kurs, som hölls den 7:e december, hölls alldeles för sent. 
Första torsdagen under andra läsperioden anses vara ett lämpligt datum, eftersom det redan är väldigt mycket som pågår under första läsperioden. 
August kontaktar Edvin Åkerfeldt och John Elfberg som höll i förra iterationen av kursen för att se ifall de kan tänka sig hålla i den även i år. 



\punkt{LUDDHack} 
Langruppen, tillsammans med XP-el, arrangerar under kommande helg 14-16:e september evenemanget "LUDDHack''. Hela styrelsen utom Frida säger sig ha för avsikt att medverka under evenemanget.

\punkt{Övriga frågor}
% Icke beslutsfattande frågor och funderingar kring föreningen.
\subpunkt{Programmeringsstugor}
Sedan tidigare har det bestämts att LUDD hålla i programmeringsstugor under torsdagar.  
Torsdag den 13:e september kommer LUDD ha programmeringsstuga. Samtidigt har datateknik-ettorna labbpass i Python. Fredrik berättar att Johan Jatko har tagit på sig att lägga upp ett Facebookevent i förväg och det föreslås att även posters bör sättas upp i förväg. 



\end{document}
