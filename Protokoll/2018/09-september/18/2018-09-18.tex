\documentclass{protokoll}

\usepackage[utf8]{inputenc}

\bild{logo.pdf}     % Ändra om man vill ha en annan bild på framsidan.
\plats{T1}          % Plats som mötet hålls i
\typ{Styrelsemöte}  %Typ av möte: Styrelsemöte|Årsmöte|medlemsmöte etc.
\datum{2018-09-18}  %Datum, form YYYY-MM-DD
\tid{18:02}{19:17}   %Tid: {START}{SLUT}.
\organ{Styrelsen}   %Det organ som har hand om mötet: Styrelsen|Mötet|Medlemmarna
\organisation{Luleå Universitets Datorförening}
\forordf{August Eriksson} 	            % Föreningsordförande
\ordf{August Eriksson}%[titel] 	    % Mötesordförande  | (titel)
\sekr{Jens Lindholm}%[titel] 		% Mötessekreterare | (titel)
\justA{Fredrik Pettersson}{Kassör} 		% Justerare A      | titel
\justB{Frida Mikkilä}{Vice Ordförande} 		% Justerare B      | titel

% Lista över övriga personer som närvarade vid mötet.
% Om det anses för många vid ett möte skriv: \person{<ANTAL> övriga medlemmar.}{}
\person{Emil Kitti}{Huvudsystemansvarig}
\person{Anton Johansson}{Ledamot}
\person{Edvin Åkerström}{Ledamot}
\person{Oscar Brink}{Valberedning}
\person{Edvin Åkerfeldt}{Revisor}
%\person{}{Medlem}
%\person{}{Medlem}

%\adjung{} % Adjungerad person
%\adjung{} % Nästa adjungerade person
%\behorighet{Trump} % Om mötet är obehörigt, inkludera denna rad med anledning(-arna).

\begin{document}
% Punkter som hanteras automagiskt:
% Mötets högtidliga öppnande - när mötet öppnats
% Formalia
%       Val av mötesordförande - den som valdes tills mötesordförande
%       Val av mötessekreterare - den som valdes till mötessekreterare (alltså du)
%       Val avtvå justeringspersoner tillika rösträknare - de som valdes till protokolljusterare
%       Mötets behöriga utlysande - om mötet är behörigt är denna automagisk annars använd \behorighet
%       (Eventuella adjungeringar) - en samling av de som inadjungerats; kontrolleras via \adjung kommadot
% Mötets högtidliga avslutande - när mötet avslutats
\newpage  


\punkt{Godkännande av dagordning}
% Skriv eventuella ändringar i dagordningen och ändra beslutet.
\begin{beslut}
     \att fastslå dagordningen som skickats ut.
\end{beslut}

\punkt{Bordlagda ärenden}
% Har det bordlagts ärenden från förra mötet?
Inga bordlagda ärenden.

\punkt{Utse ansvarig(a) för NCPC}
Edvin Åkerfeldt tar på sig ansvaret för LUDDs medverkan i NCPC.

\punkt{Uppdateringar av pågående projekt}
% Vad har föreningen gjort med pågående projekt?
\subpunkt{LEDs i korridor}
En del av en av LUDDs LED-slingor behöver lödas och ett nytt nätaggregat behöver köpas in då det existerande är defekt eller har för låg effekt. 

\subpunkt{Design av banner till LUDD}
Adam Sahlin har kontaktats av August och vill ej vara ansvarig för någon design. Diskussion kring ny designer och nytt ansvar hålls. 
Fredrik tar på sig ansvaret för att ta fram designförslag till nästa möte. 

\subpunkt{Ny statisk webbsida}
Hemsidan är horribel. Fredrik och Emil har kollat mer på hemsidan och kommit fram till att den är skit och att den bör bytas. 

\subpunkt{Övriga projekt}
\subsubsection{LED-skylt}
LED-skylten har stått mörk sedan LUDDHack då LED-strips:en och tillhörande nätagg och kontroller gick sönder under eventet. 
\begin{beslut}
    \att avsätta 500kr för inköp av nya LED-strips och nätaggregat.
\end{beslut}

\subsubsection{Koldioxidsensor}
Edvin Åkerström har tagit fram tre prisförslag. Beslut om vilket som ska ageras på bedöms kunna tas utanför mötet. 

\subsubsection{LUDDTröjor}
LUDD vill ha tröjor med LUDD-designs. Diskussion kring allmän design och ansvarig för att design tas fram bryter ut. 
Edvin Åkerström tar på sig ansvaret för att ta fram designförslag. 
New Profile har ett samarbete med Teknologkåren som är väldigt attraktivt. Mötet ser något liknande som väldigt intressant.
August tar på sig ansvaret att ringa och kontakta företaget.

\punkt{Beslut om inköp}
\subpunkt{Stativ för VR}
Jens har tagit fram ett prisförslag för inköp av stativ. Mötet anser att beslut kring inköp inte är brådskande. 
\begin{beslut}
    \att bordlägga ärendet till nästa möte.
\end{beslut}
\subpunkt{Telefonladdare med sladdar}
LUDD skulle behöva fler sladdar till laddarna i lokalen. 
\begin{beslut}
    \att avsätta 250kr från lokalbudgeten för att köpa micro usb/lightning/usb-c sladdar.
\end{beslut}

\punkt{Styrelsens ställning gällande Langruppen}
I princip ingen i Langruppen är aktiv exklusive styrelsen och de som .
Styrelsen anser att Langruppen inte fyller någon roll längre och bör avvecklas.

\punkt{Problem och dispyter som uppstått sedan Nolleperioden}
Vissa nya medlemmar som gått med under nolleperioden har tagit med kamrater som inte är LUDDmedlemmar och låtit dessa ta kaffe, skriva ut och använda LUDD som studieyta gratis vilket orsakat problem med trängsel, utgifter och nedskräpning samt ett influx av icke-medlemmar och icke seriösa medlemmar. 
Lånekoppar och pappersmuggar skräpar och mötet anser dessa vara överflödiga, då aktiva medlemmar bedöms kunna ha egna koppar. Styrelsen har även fått rapporter om att en viss medlem, Josef Utbult, förstör andra medlemmars koppar.
En lägre och faktiskt påtvingad maximal mängd utskrifter per tidsperiod införs i ordningsreglerna under punkt §4.4.5. Den nuvarande gränsen på 100 utskrifter per vecka föreslås ändras till 100 utskrifter per månad. 
\begin{beslut}
    \att ändra ordet "vecka" till "månad" under §4.4.5 i stadgarna samt tillämpa reglerna.
\end{beslut}
\punkt{Övriga frågor}
% Icke beslutsfattande frågor och funderingar kring föreningen.
Inga övriga frågor.


\end{document}
