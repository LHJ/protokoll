\documentclass{protokoll}

\usepackage[utf8]{inputenc}

\bild{logo.pdf}     % Ändra om man vill ha en annan bild på framsidan.
\plats{T1}          % Plats som mötet hålls i
\typ{Styrelsemöte}  %Typ av möte: Styrelsemöte|Årsmöte|medlemsmöte etc.
\datum{2018-08-29}  %Datum, form YYYY-MM-DD
\tid{18:02}{20:03}   %Tid: {START}{SLUT}.
\organ{Styrelsen}   %Det organ som har hand om mötet: Styrelsen|Mötet|Medlemmarna
\organisation{Luleå Universitets Datorförening}
\forordf{August Eriksson} 	            % Föreningsordförande
\ordf{August Eriksson}%[titel] 	    % Mötesordförande  | (titel)
\sekr{Jens Lindholm}%[titel] 		% Mötessekreterare | (titel)
\justA{Edvin Åkerström}{Ledamot} 		% Justerare A      | titel
\justB{Anton Johansson}{Ledamot} 		% Justerare B      | titel

% Lista över övriga personer som närvarade vid mötet.
% Om det anses för många vid ett möte skriv: \person{<ANTAL> övriga medlemmar.}{}
\person{Fredrik Pettersson}{Kassör}

%\adjung{} % Adjungerad person
%\adjung{} % Nästa adjungerade person
%\behorighet{Trump} % Om mötet är obehörigt, inkludera denna rad med anledning(-arna).

\begin{document}
% Punkter som hanteras automagiskt:
% Mötets högtidliga öppnande - när mötet öppnats
% Formalia
%       Val av mötesordförande - den som valdes tills mötesordförande
%       Val av mötessekreterare - den som valdes till mötessekreterare (alltså du)
%       Val avtvå justeringspersoner tillika rösträknare - de som valdes till protokolljusterare
%       Mötets behöriga utlysande - om mötet är behörigt är denna automagisk annars använd \behorighet
%       (Eventuella adjungeringar) - en samling av de som inadjungerats; kontrolleras via \adjung kommadot
% Mötets högtidliga avslutande - när mötet avslutats
\newpage  


\punkt{Godkännande av dagordning}
% Skriv eventuella ändringar i dagordningen och ändra beslutet.
\begin{beslut}
     \att fastslå dagordningen som skickats ut.
\end{beslut}

\punkt{Bordlagda ärenden}
% Har det bordlagts ärenden från förra mötet?
Inga bordlagda ärenden.

\punkt{Nolleperioden}
% Skriv det som är relevant kring punkten.
Inför överlämning till nästa styrelse ska nuvarande styrelsen informera om vikten av förberedelse för rundvandringen under Nolleperioden. 

Styrelsen anser att spelkvällen gick bra med avseende på underhållning, nya medlemmar och PR.

Sektionsdagens gyckel var populärt under tiden det pågick. Eftereffekterna, som inkluderat sjukhusvisit för vissa medverkande, var mindre uppskattade och ledde till att Nolleperiodsgruppen har förbjudit stark sås från framtida gyckel. 

Öppet hus-dagen och den tillhörande Luddolympiaden var lyckad. 

\punkt{Pågående projekt}

\subpunkt{LEDs i korridor}
Inget att rapportera.

\subpunkt{Profilering}
August har tagit fram ett prisförslag för en banderoll på cirka 1000kr från vistaprint.se. Styrelsen ser positivt till inköpet.
\begin{beslut}
    \att allokera 1500kr från budgetposten Intern Representation för att köpa banderollen.
\end{beslut}
En design måste också tas fram innan köpet kan slutföras. August tar på sig ansvaret för att en design ska vara färdig till nästa möte.

\subpunkt{Ny statisk hemsida}
Fredrik och Emil Kitti har kollat på den existerande hemsidan och dess mall, byte av mall övervägs då den som används bedöms inte var särskilt användarvänlig.

\subpunkt{Övriga projekt}
Inget annat att rapportera.

\punkt{Ny boschansvarig}
Emil Kitti är inaktiv som Boschansvarig och en ny behöver väljas. Anton har periodvis tillgång till att låna bil och kan tänka sig vara andra Boschansvarig tillsammans med August. 

\punkt{Inköp}

\subpunkt{Ny frys}
Inköp till Bosch måste i dagsläget göras oftare än vad ansvariga skulle önska, huvudsakligen då nuvarande frys inte rymmer lika mycket som går åt under perioden som man inte köper in. En lösning som framförts är att ha en frys för lagring av extra frysvaror mellan inköp. Styrelsen utforskar möjligheter på Blocket. 
\begin{beslut}
    \att allokera 1000kr från budgetposten Lokal för att köpa en ny frys.
\end{beslut}

\subpunkt{Soppåsar}
Soppåsar behöver införskaffas. Detta kommer att göras antingen hos Snabbgross eller godtycklig byggaffär.

\subpunkt{Färg till målning}
Vit samt eventuellt kompletterande färg behöver köpas in. 

\subpunkt{Övrigt}

\subsubsection{Projektorlampa}
Efter att projektorns lampa senast gick sönder har inget extraexemplar köpts in, något som bedöms bör göras för att undvika driftstopp. 
\begin{beslut}
    \att allokera 1000kr från budgetposten Lokal för inköp av en extra projektorlampa.
\end{beslut}

\subsubsection{Koldioxidsensor}
Nödvändighet och användbarhet av en koldioxidmätare diskuteras, detta samt pris av en ny koldioxidmätare bedöms vara tillräckligt för att medlemmar ska vara med och bestämma om inte styrelsen lyckas få tag på en gratis. 
\begin{beslut}
    \att bordlägga ärendet till nästa medlemsmöte.
\end{beslut}

\punkt{Målning av logga utomhus}
Målning har inte blivit av på grund av väderförhållanden. Styrelsen planerar att måla ett nytt märke för LUDD vid bron så snart vädret tillåter.

\punkt{Övriga ärenden}
\subpunkt{Griffeltavla}
Då LUDD:s griffeltavla blivit vandaliserad av Valvets gäster under Nolleperioden behöver den ersättas med en ny. Valvet har tagit på sig ansvaret att ersätta LUDD med en ny, varefter styrelsen måste ta fram ett prisförslag. 
Edvin tar på sig att ta fram detta prisförslag. 

% Icke beslutsfattande frågor och funderingar kring föreningen.
\subpunkt{Huvudsystemansvarig avgår}
Marcus Lund avgår från sin post som huvudsystemansvarig med anledning av flytt till Stockholm. Styrelsen och rootgruppen ska arbeta för att finna en ersättare snarast. 

\subpunkt{Prioritering av GitLab issues}
Styrelsen gick igenom och uppdaterade samtliga pågående projekt på dess GitLab issue tracker.

\end{document}
