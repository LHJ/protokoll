\documentclass{protokoll}

\usepackage[utf8]{inputenc}

\bild{logo.pdf}     % Ändra om man vill ha en annan bild på framsidan.
\plats{T1}          % Plats som mötet hålls i
\typ{Styrelsemöte}  %Typ av möte: Styrelsemöte|Årsmöte|medlemsmöte etc.
\datum{2018-08-16}  %Datum, form YYYY-MM-DD
\tid{18:09}{19:41}   %Tid: {START}{SLUT}.
\organ{Styrelsen}   %Det organ som har hand om mötet: Styrelsen|Mötet|Medlemmarna
\organisation{Luleå Universitets Datorförening}
\forordf{August Eriksson} 	            % Föreningsordförande
\ordf{August Eriksson}%[titel] 	    % Mötesordförande  | (titel)
\sekr{Jens Lindholm}%[Ledamot] 		% Mötessekreterare | (titel)
\justA{Anton Johansson}{Ledamot} 		% Justerare A      | titel
\justB{Edvin Åkerström}{Ledamot} 		% Justerare B      | titel

% Lista över övriga personer som närvarade vid mötet.
% Om det anses för många vid ett möte skriv: \person{<ANTAL> övriga medlemmar.}{}
\person{Frida Mikkilä}{Vice Ordförande}
\person{Jens Lindholm}{Sekreterare}
%\person{Anton Johansson}{Ledamot}
\person{Fredrik Petterson}{Kassör}
\person{Johan Jatko}{Root}
%\person{Edvin Åkerström}{Ledamot}
%\adjung{Axel Lund} % Adjungerad person
%\adjung{asd} % Nästa adjungerade person
%\adjung{potat}
%\adjung{Mitt Liv}
%\behorighet{Trump} % Om mötet är obehörigt, inkludera denna rad med anledning(-arna).

\begin{document}
% Punkter som hanteras automagiskt:
% Mötets högtidliga öppnande - när mötet öppnats
% Formalia
%       Val av mötesordförande - den som valdes tills mötesordförande
%       Val av mötessekreterare - den som valdes till mötessekreterare (alltså du)
%       Val avtvå justeringspersoner tillika rösträknare - de som valdes till protokolljusterare
%       Mötets behöriga utlysande - om mötet är behörigt är denna automagisk annars använd \behorighet
%       (Eventuella adjungeringar) - en samling av de som inadjungerats; kontrolleras via \adjung kommadot
% Mötets högtidliga avslutande - när mötet avslutats
\newpage  


\punkt{Godkännande av dagordning}
% Skriv eventuella ändringar i dagordningen och ändra beslutet.
\begin{beslut}
     \att fastlå den utskickade dagordningen med undantag av de två punkterna ''Bordlagda ärenden'' och ''Pågående projekt''.
\end{beslut}

\punkt{Påstigning som ny styrelse}
Fredrik har inte fått någon överlämning som kassör, tidigare kassör Edvin Åkerfeldt ska återvända till helgen och hoppas kunna ge sådan snarast. 

\subpunkt{Rutiner för verksamhetsår}
Mötet diskuterar hur ofta möten behövs, Fredrik föreslår att Emil Kitti's föreslag med inofficiella diskussionsmöten omlott med riktiga beslutsmöten. Mötet kommer fram till att diskussion mest troligt kommer saknas från protokollen om detta skulle bli rutin. 
Diskussion kring flytt av dag för styrelsemöten bryter ut. 
Mötet beslutar att styrelsemöten ska hållas varje vecka, som tidigare, men på tisdagar. 

\subpunkt{Val av ny PR-ansvarig}
Edvin Åkerström tar på sig att vara ny PR-ansvarig.
\punkt{Pågående projekt}
\subpunkt{Trappa}
Johan tar på sig att designa och bygga en trappa om material köps in och han kontaktas när detta gjorts. Detta skall göras av August någon gång i september.
\subpunkt{Bardisk}
Johan tar på sig att designa och bygga en bardisk, med hjälp, om material köps in och han kontaktas när detta är gjort. Johan uppskattar arbetstiden som krävs till att vara circa en helg. August ska kontakta Johan vid någon punkt i framtiden.

\subpunkt{Profilering}
Ludd skulle vilja ha en banner att hänga vid plattformen till spelkvällen, även om vi hade en design nu hade vi nog inte hunnit beställa och få det levererat i tid.

\subpunkt{Varuautomat}
Varuautomaten som står vid BOSCH bedöms vara i vägen, när den är igång låter dess kompressor väldigt mycket och den är därför avstängd. Automaten ska flyttas till korridoren, kontrollen digitaliseras och sedan användas för att sälja exempelvis Raspberry Pi:s.

\subpunkt{Ny statisk hemsida}
Ludd har en hemsida som varit under konstruktion sen länge. Fredrik kan tänka sig kolla på hur den är byggd och, beroende på hur mycket jobb det skulle vara, eventuellt bygga klart och fylla ut den med information.

\subpunkt{Trasiga LEDs i korridor}
Närvarande vid mötet är osäkra på vad exakt det är för problem med LED-belysningen. 
Anton och Jens tar på sig att undersöka hur det ser ut. 


\subpunkt{Övriga projekt}
Fredrik tycker att sitt projekt ska läggas ner och alla delar återvinnas då han känner att det finns andra lösningar som åstadkommer samma mål fast bättre.

\subpunkt{Målning av rutan}
Den ruta med Vicke som målades strax innan bron vid Slemmis förra året syntes dåligt och skulle behövas målas om, men denna gång med vit färg som bakgrund för extra synlighet. 
Vit färg kommer att köpas in samt eventuell extrafärg om den svarta och röda som vart över inte räcker. 

\subpunkt{Målning av vickeväggen}
\begin{beslut}
    \att bordlägga punkten till nästa möte.
\end{beslut}

\subpunkt{LED-skylt}
Ludd har fått en hemgjord LED-skylt donerad till sig.  
Edvin Åkerström tar på sig att fixa arrangerandet av LED:s och kontrollern.

\punkt{Ny frys/frysbox}
En frysbox skulle vara väldigt användbar för att kunna ta in större mängder frysvaror mindre ofta. Pengar bedöms kunna tas från budgetposten BOSCH. 

\punkt{Bankärenden}
Nuvarande styrelsen har inte tillgång till Swedbanks internetbank eftersom förra styrelsen inte har skrivit och rättat klart årsmötesprotokollet där de valdes in.

\punkt{Nolleperioden}
\subpunkt{Rundvandringen 22/8}
Efter diskussion kommer styrelsen fram till att den ska vara på Ludd, ha dörren öppen och spela lite musik.

\subpunkt{Öppet hus}
Styrelsen ska hålla i de aktiviteter som vanligtvis hålls under Luddolympiaden. 

\subpunkt{Spelkvällen 26/8}
En inventarielista av Ludds konsoller, TV-apparater och liknande, samt saker som ska utlånas, måste skrivas innan Spelkvällen. 
Frida tar på sig ansvar för att det blir gjort. 

\subpunkt{Nollemässan 27/8}
Ludd ska vara närvarande, och vill ha någon form av tävling. Det föreslås att tävlingen kan gå ut på att nollor exempelvis får se en bild från serverhallen och gissa hur många av något finns. 
Styrelsen ska organisera och ordna ett schema över Telegram. 

\subpunkt{Gyckel på datasittningen 24/8}
Chicken race med burn. Ett antal personer kämpar för att äta upp en mängd burn med en mängd nachochips utan att dricka från sitt glas mjölk som placerats framför dem. Vinnande klass får en stor förpackning glass som efterrätt, vilket förväntas kunna vara väldigt attraktivt för nollorna då de inte serveras någon efterrätt annars under sittningen.
Samtidigt får vi reklam för Tacophaesten.

\punkt{Målning av labbrockar}
Styrelsen ska ta itu med detta klockan 13:00 den 18:e august.

\punkt{Övriga ärenden}
Inga övriga ärenden.


\end{document}
