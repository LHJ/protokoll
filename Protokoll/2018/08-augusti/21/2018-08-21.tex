\documentclass{protokoll}

\usepackage[utf8]{inputenc}

\bild{logo.pdf}     % Ändra om man vill ha en annan bild på framsidan.
\plats{T1}          % Plats som mötet hålls i
\typ{Styrelsemöte}  %Typ av möte: Styrelsemöte|Årsmöte|medlemsmöte etc.
\datum{2018-08-21}  %Datum, form YYYY-MM-DD
\tid{16:06}{16:47}   %Tid: {START}{SLUT}.
\organ{Styrelsen}   %Det organ som har hand om mötet: Styrelsen|Mötet|Medlemmarna
\organisation{Luleå Universitets Datorförening}
\forordf{August Eriksson} 	            % Föreningsordförande
\ordf{August Eriksson}%[titel] 	    % Mötesordförande  | (titel)
\sekr{Jens Lindholm}%[titel] 		% Mötessekreterare | (titel)
\justA{Frida Mikkilä}{Vice Ordförande} 		% Justerare A      | titel
\justB{Fredrik Pettersson}{Kassör} 		% Justerare B      | titel

% Lista över övriga personer som närvarade vid mötet.
% Om det anses för många vid ett möte skriv: \person{<ANTAL> övriga medlemmar.}{}
\person{Frida Mikkilä}{Vice Ordförande}
\person{Fredrik Petterson}{Kassör}
\person{Anton Johansson}{Ledamot}
\person{Edvin Åkerström}{Ledamot}

\person{Niklas Ulfvarson}{Medlem}
\person{Erik Viklund}{Medlem}
\person{Johan Jatko}
%\adjung{} % Adjungerad person
%\adjung{} % Nästa adjungerade person
%\behorighet{Trump} % Om mötet är obehörigt, inkludera denna rad med anledning(-arna).

\begin{document}
% Punkter som hanteras automagiskt:
% Mötets högtidliga öppnande - när mötet öppnats
% Formalia
%       Val av mötesordförande - den som valdes tills mötesordförande
%       Val av mötessekreterare - den som valdes till mötessekreterare (alltså du)
%       Val avtvå justeringspersoner tillika rösträknare - de som valdes till protokolljusterare
%       Mötets behöriga utlysande - om mötet är behörigt är denna automagisk annars använd \behorighet
%       (Eventuella adjungeringar) - en samling av de som inadjungerats; kontrolleras via \adjung kommadot
% Mötets högtidliga avslutande - när mötet avslutats
\newpage  


\punkt{Godkännande av dagordning}
% Skriv eventuella ändringar i dagordningen och ändra beslutet.
\begin{beslut}
     \att fastslå dagordningen som skickats ut.
\end{beslut}

\punkt{Nolleperioden}
\subpunkt{Nollemässan}
Mässan kommer hållas från klockan 09:00 måndagen den 28:e augusti. Schema kommer organiseras över Telegram och Doodle.

\subpunkt{Rundvandringen}
Rundvandringen kommer hållas den 22:a augusti från klockan 13:30. Ludd kommer ha folk på plats från lunch och frammåt.

\subpunkt{Öppet hus/Luddolympiaden}
Luddolympiaden ska hållas den 24:e augusti. Pris ska bestämmas över Telegram.

\subpunkt{Gyckel på datasittning}
Datasittningen ska hållas på kvällen den 24:e augusti, styrelsen har kontaktats och frågats om Ludd ska hålla i något gyckel. Styrelsen ser positivt till denna idé och diskuterar möjliga gyckel. Preliminärt gyckel bestäms vara någon form av burnätartävling.  

Ideér för omstart diskuteras. Omstart bestäms vara maränger till glassen samt eventuellt en sombrero till publiken.

\subpunkt{Spelkvällen}
Huruvida att LURC:s 200 utbytesstudenter eventuellt kommer överväldiga spelkvällen diskuteras. Möjliga lösningar, såsom att man ber NPG tar dit dom lite senare, diskuteras. August tar på sig att kontakta NPG.  

Fredrik tar på sig att vara Boschansvarig hela dagen.

\punkt{Pullupbar}
% Skriv det som är relevant kring punkten.
Fredrik vill fortfarande ha en pullupbar i korridoren. Det har uppförts en crowdfundinglista för betalningen av själva stången. 
Mötet bifaller Fredriks förfrågan.

\punkt{Gitlab issue tracker}
Fredrik presenterar Ludds issue tracker och hur den fungerar för resten av styrelsen samt föreslår att den ska börja användas mer aktivt. Styrelsen ser positivt på detta och bestämmer att den ska användas och påminnas om mer aktivt. 

\punkt{Övriga ärenden}
% Icke beslutsfattande frågor och funderingar kring föreningen.
Bosch behöver fyllas på och lokalen behöver städas om innan Nolleperioden börjar. Lokalen ska städas efter mötet och Bosch ska fyllas på inom den närmsta framtiden.


\end{document}
