\documentclass{protokoll}

\usepackage[utf8]{inputenc}

\bild{logo.pdf}     % Ändra om man vill ha en annan bild på framsidan.
\plats{T1}          % Plats som mötet hålls i
\typ{Styrelsemöte}  %Typ av möte: Styrelsemöte|Årsmöte|medlemsmöte etc.
\datum{2019-02-17}  %Datum, form YYYY-MM-DD
\tid{15:13}{16:15}   %Tid: {START}{SLUT}.
\organ{Styrelsen}   %Det organ som har hand om mötet: Styrelsen|Mötet|Medlemmarna
\organisation{Luleå Universitets Datorförening}
\forordf{August Eriksson} 	            % Föreningsordförande
\ordf{August Eriksson}%[titel] 	    % Mötesordförande  | (titel)
\sekr{Jens Lindholm}%[titel] 		% Mötessekreterare | (titel)
\justA{Edvin Åkerström}{Ledamot} 		% Justerare A      | titel
\justB{Anton Johansson}{Ledamot} 		% Justerare B      | titel

% Lista över övriga personer som närvarade vid mötet.
% Om det anses för många vid ett möte skriv: \person{<ANTAL> övriga medlemmar.}{}
%\person{Emil Kitti}{Huvudsystemansvarig}
\person{Fredrik Pettersson}{Kassör}
\person{Aran Nur}{Medlem}
\person{Marcus Eriksson}{Medlem}
%\adjung{} % Adjungerad person
%\adjung{} % Nästa adjungerade person
%\behorighet{Trump} % Om mötet är obehörigt, inkludera denna rad med anledning(-arna).


\begin{document}
% Punkter som hanteras automagiskt:
% Mötets högtidliga öppnande - när mötet öppnats
% Formalia
%       Val av mötesordförande - den som valdes tills mötesordförande
%       Val av mötessekreterare - den som valdes till mötessekreterare (alltså du)
%       Val avtvå justeringspersoner tillika rösträknare - de som valdes till protokolljusterare
%       Mötets behöriga utlysande - om mötet är behörigt är denna automagisk annars använd \behorighet
%       (Eventuella adjungeringar) - en samling av de som inadjungerats; kontrolleras via \adjung kommadot
% Mötets högtidliga avslutande - när mötet avslutats
\newpage  


\punkt{Godkännande av dagordning}
% Skriv eventuella ändringar i dagordningen och ändra beslutet.
\begin{beslut}
     \att fastslå dagordningen som skickats ut.
\end{beslut}

\punkt{Bordlagda ärenden}
% Har det bordlagts ärenden från förra mötet?
Inga bordlagda ärenden.

\punkt{Behandling av Roots motion för inköp av grafikkort}
August presenterar en motion inskickad av Johan Jatko på roots vägnar. Se
bilaga för motion. 
\begin{beslut}
    \att bifalla motionen och allokera 10000kr från budgetposten ''Projekt'' till detta.
\end{beslut}

\punkt{Diskussion och ev.\ beslut avseende kränkande betéende i LUDD:s lokaler}
% Skriv det som är relevant kring punkten.
Styrelsen har fått klagomål att medlemmar begår eventuella brott mot LTU:s 
beteenderegler, August har föreslagit ett tillägg till ordningsreglerna.  

Diverse tolkningar av LUDD:s och LTU:s regler diskuteras för hur detta tillägg
ska formuleras. 

\begin{beslut}
  \att Införa den föreslagna punkten ''\S 4.4.6. Samtalsklimat'' under ''\S
  4.4. Ordningsregler'' i styrdokumentet. 
\end{beslut}
Punkten i sin helhet lyder: 
''4.4.6. Samtalsklimat


Ingen ska känna sig ovälkommen i LUDDs lokaler och verksamhet, oavsett ursprung, religion och dylikt. Den som uttrycker sig på ett sätt som bryter mot följande samtalsregler riskerar att bli utesluten ur föreningen med omedelbar verkan.

\begin{itemize}
  \item Man får inte uttala sig på ett sätt som medvetet upplevs kränkande av andra medlemmar i föreningen.
  \item Medlemmar ska i LUDDs lokaler bete sig i enlighet med LTUs regler, bland annat https://www.ltu.se/student/Planera/Mina-rattigheter-och-skyldigheter/Likabehandling-for-studenter-1.7749''
\end{itemize}

\punkt{Övriga ärenden}
% Icke beslutsfattande frågor och funderingar kring föreningen.
Inga övriga ärenden.

\end{document}
