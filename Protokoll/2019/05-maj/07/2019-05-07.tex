\documentclass{protokoll}

\usepackage[utf8]{inputenc}

\bild{logo.pdf}     % Ändra om man vill ha en annan bild på framsidan.
\plats{A3024}          % Plats som mötet hålls i
\typ{Styrelsemöte}  %Typ av möte: Styrelsemöte|Årsmöte|medlemsmöte etc.
\datum{2019-05-07}  %Datum, form YYYY-MM-DD
\tid{16:37}{17:57}   %Tid: {START}{SLUT}.
\organ{Styrelsen}   %Det organ som har hand om mötet: Styrelsen|Mötet|Medlemmarna
\organisation{Luleå Universitets Datorförening}
\forordf{August Eriksson} 	            % Föreningsordförande
\ordf{Anton Johansson}[Ledamot]%[titel] 	    % Mötesordförande  | (titel)
\sekr{Jens Lindholm}%[titel] 		% Mötessekreterare | (titel)
\justA{Oscar Brink}{Medlem} 		% Justerare A      | titel
\justB{Samuel Gradén}{Medlem} 		% Justerare B      | titel

% Lista över övriga personer som närvarade vid mötet.
% Om det anses för många vid ett möte skriv: \person{<ANTAL> övriga medlemmar.}{}
\person{Emil Kitti}{Huvudsystemansvarig}
\person{Edvin Åkerström}{Ledamot}

\person{Louise Sehlstedt}{Medlem}
%\person{}{}
%\person{}{}
%\adjung{} % Adjungerad person
%\adjung{} % Nästa adjungerade person
%\behorighet{Trump} % Om mötet är obehörigt, inkludera denna rad med anledning(-arna).

\begin{document}
% Punkter som hanteras automagiskt:
% Mötets högtidliga öppnande - när mötet öppnats
% Formalia
%       Val av mötesordförande - den som valdes tills mötesordförande
%       Val av mötessekreterare - den som valdes till mötessekreterare (alltså du)
%       Val avtvå justeringspersoner tillika rösträknare - de som valdes till protokolljusterare
%       Mötets behöriga utlysande - om mötet är behörigt är denna automagisk annars använd \behorighet
%       (Eventuella adjungeringar) - en samling av de som inadjungerats; kontrolleras via \adjung kommadot
% Mötets högtidliga avslutande - när mötet avslutats
\newpage  


\punkt{Godkännande av dagordning}
% Skriv eventuella ändringar i dagordningen och ändra beslutet.
\begin{beslut}
     \att fastslå dagordningen som skickats ut med tillägg av punkten
     ''Diskussion kring ställningstaganden för LTU:s riktlinjer kring
     akademiska studier'' efter \S 8.
\end{beslut}

\punkt{Bordlagda ärenden}
% Har det bordlagts ärenden från förra mötet?
Inga bordlagda ärenden.

\punkt{Uppdateringar av pågående projekt}
\subpunkt{LUDD-tröjor, LUDD-banner}
Inget att uppdatera. 

\subpunkt{Koldioxidsensor}
Edvin framför att det planeras arbetas mer på projektet under sommaren som
följer. 

\subpunkt{Övriga projekt}
Inga övriga projekt. 

\punkt{Återställande av styrdokument, uppdaterade stadgar och övrig
dokumentation som förlorats i samband med intrång på LUDDs Gitlab-server}
Efter att ett administratörskonto blev hackat och alla projekt under ''ludd''-
och ''styrelse''-grupperna blivit borttagna från LUDD:s GitLab server har bland
annat ''LUDD:s samlade styrdokument''- och ''protokoll''-repot

August och Jens är tillsammans ansvariga för att återställa ''LUDD:s samlade 
styrdokument'' och Jens för att återställa ''protokoll'' så gott som det går. 

root-gruppen är ansvariga för att informera medlemmarna om säkerhetsintrånget
och vilka åtgärder som följer.

\punkt{Införande av pris på LUDDs kaffe}
 \subpunkt{Diskussion huruvida kaffe inte längre bör vara gratis}
 Diskussion kring ifall det är nödvändigt eller fördelaktigt att börja ta
 betalt för kaffe från LUDD:s kaffemaskin förs. 


 Mötet är ense om att kaffe bör kosta en symbolisk summa, delvis för att täcka
 kostnader och delvis för att uppmuntra till användning av Bosch. 


Mötet beslutar att preliminärt sätta priset per kopp till 473 Boschmoney,
motsvarande 47,3 öre.
 \subpunkt{Diskussion kring pågående projekt att sammankoppla kaffemaskin med
 Bosch}
 Systemet har karaktäriserats men en hel del arbete krävs för att det ska kunna
 genomföras, så även utrustning.
 \subpunkt{Beslut att avsätta 700 SEK från projektbudget för inköp av materiell
 för att sammankoppla kaffemaskin med Bosch-system} 
\begin{beslut}
  \att avsätta 700kr från projektbudgeten till utrustning för att sammankoppla
  kaffemaskinen och Bosch.
\end{beslut}
\punkt{Blivande styrelsens planering och förberedelse inför starten på
nästa verksamhetsår}
Blivande styrelsen har en inplanerad gemenskapsdag under söndagen som följer,
12/5, under vilken planer för verksamhetsåret kommer diskuteras. 

I och med brädspelsföreningen BoardGameDungeon:s upplösande kommer mest troligt 
inte Spelkvällen, som tidigare hållits i under Nolleperioden tillsammans med 
LUDD, hållas i på samma sätt som tidigare. Nolleperioden har inte kontaktat
LUDD med någon information inför höstens Nolleperiod.


\punkt{Diskussion gällande förslag att införa ett s.k. ''bounty system''}
Samuel framför att Neava uttryckte intresse för att LUDD:s medlemmar skulle
kunna få arbeta på mindre projekt på beställning åt Neava. Mötet ser positivt
till att agera mellanhand och fostra relationer mellan LUDD, dess medlemmar och
intresserade företag. 

Mötet förväntas diskutera och framföra idéer kring anknytningar till
arbetslivet under ett diskutionsmöte på måndagen som följer, 13/5.

\punkt{Diskussion kring ställningstaganden för LTU:s riktlinjer kring 
akademiska studier}
Efter att en medlem har begärt svar till inlämningsuppgifter i LUDD:s
Telegram-chatt i utbyte mot betalning har huruvida detta ska betraktas som
försök till fusk samt ifall detta skulle vara grund till uteslutning från
föreningen diskuterats i och utanför samma chatt. 

Mötet är enigt om att detta inte är tillräckligt för att motivera uteslutning
och att det, efter diskussion med medlemmen i fråga, har framgått att det
huvudsakligen var en fråga om formulering.

\punkt{Ogiltigförklaring av styrelsemöte med tillhörande beslut 2019-04-30}
\begin{beslut}
  \att ogiltigförklara mötet från den 30:e april.
\end{beslut}
\punkt{Förstärkande av per capsulam-beslut att avsätta 3000 SEK till
elektriker}
Denna punkt existerar för att förankra per capsulam-beslutet. 

Ingen styrelsemedlem motsatte sig beslutet när det togs.

\punkt{Beslut gällande Roots motion att avsätta 16 000 SEK för elinstallation}
I och med att det tidigare mötet, vilket var extrainkallat för att ta beslut
för denna punkt, har ogiltigförklarats tas beslutet igen. 
\begin{beslut}
  \att avsätta 16 000 SEK från lokalbudgeten för elinstallation.
\end{beslut}

\punkt{Beslut gällande inköp av högtalare och mixerbord för 3000 SEK}
Ett set med två högtalare och ett mixerbord har köpts in.  
Diskussion kring ifall LUDD skulle ha användning för utrustningen förs. 
\begin{beslut}
  \att avsätta 3000 SEK från lokalbudgeten för inköp av högtalare och
  mixerbord. 
\end{beslut}

\punkt{Beslut gällande inköp av fortbildningsmaterial för framtida styrelser
för 660 SEK}

\begin{beslut}
  \att avsätta 660 SEK från budgetposten ''utbildning'' för inköp av
  fortildningsmaterial.
\end{beslut}

\punkt{Övriga frågor}
% Icke beslutsfattande frågor och funderingar kring föreningen.
Inga övriga frågor.


\end{document}
