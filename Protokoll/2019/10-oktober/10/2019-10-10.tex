\documentclass{protokoll}

\usepackage[utf8]{inputenc}

\bild{logo.pdf}     % Ändra om man vill ha en annan bild på framsidan.
\plats{T1}          % Plats som mötet hålls i
\typ{Medlemsmöte}  %Typ av möte: Styrelsemöte|Årsmöte|medlemsmöte etc.
\datum{2019-10-07}  %Datum, form YYYY-MM-DD
\tid{18:01}{19:23}   %Tid: {START}{SLUT}.
\organ{Styrelsen}   %Det organ som har hand om mötet: Styrelsen|Mötet|Medlemmarna
\organisation{Luleå Universitets Datorförening}
\forordf{Anton Johansson} 	            % Föreningsordförande
\ordf{Anton Johansson}%[titel] 	    % Mötesordförande  | (titel)
\sekr{Jens Lindholm}%[titel] 		% Mötessekreterare | (titel)
\justA{Edvin Åkerfeldt}{Medlem} 		% Justerare A      | titel
\justB{Edvin Åkerström}{Medlem} 		% Justerare B      | titel

% Lista över övriga personer som närvarade vid mötet.
% Om det anses för många vid ett möte skriv: \person{<ANTAL> övriga medlemmar.}{}
\person{Josef Utbult}{Ledamot}
%\person{Oscar Brink}{Kassör}
\person{Samuel Gradén}{Ledamot}
\person{Louise Sehlstedt}{Vice Ordförande}
\person{August Eriksson}{}
\person{Lisa Jonsson}{}
\person{Mikael Hedkvist}{}
\person{7 andra}{}
%\person{}{}
%\person{}{}
%\person{}{}
%\person{X andra}{}

%\adjung{} % Adjungerad person
%\adjung{} % Nästa adjungerade person
%\behorighet{Trump} % Om mötet är obehörigt, inkludera denna rad med anledning(-arna).

\begin{document}
% Punkter som hanteras automagiskt:
% Mötets högtidliga öppnande - när mötet öppnats
% Formalia
%       Val av mötesordförande - den som valdes tills mötesordförande
%       Val av mötessekreterare - den som valdes till mötessekreterare (alltså du)
%       Val avtvå justeringspersoner tillika rösträknare - de som valdes till protokolljusterare
%       Mötets behöriga utlysande - om mötet är behörigt är denna automagisk annars använd \behorighet
%       (Eventuella adjungeringar) - en samling av de som inadjungerats; kontrolleras via \adjung kommadot
% Mötets högtidliga avslutande - när mötet avslutats
\newpage  


\punkt{Godkännande av dagordning}
% Skriv eventuella ändringar i dagordningen och ändra beslutet.
\begin{beslut}
     \att fastslå dagordningen som skickats ut.
\end{beslut}


\punkt{Rapport från styrelsen}
\subpunkt{Presentation av verksamhetsplan för verksamhetsåret 2019/2020}
Anton presenterar styrelsens verksamhetsplan 2019/2020. 

\subpunkt{Presentation av nya valda funktionärer av styrelsen sedan föregående föreningsmöte}
Styrelsen har utsett två personer, Lisa Jonsson och Jonas Jacobsson, till 
posten Valberedning, varav Lisa sammankallande. Jonas kunde ej medverka på
grund av sjukdom. 

Lisa presenterar sig för mötet. 


\punkt{Rapport från revisor}
August presenterar en positiv muntlig rapport.
\punkt{Rapport från styrelsen 2018/2019}
\subpunkt{Presentation och diskussion kring föregående styrelsens verksamhetsberättelse}

August har skickat in en rapport från föregående styrelse, Anton tar ordet och 
presenterar denna. Se bilaga.

\subpunkt{Presentation och diskussion kring föregående styrelsens
  kassaberättelse}
Fredrik Pettersson tar ordet och presenterar status för LUDD:s ekonomi. 
Se bilagor.


Mikael omber styrelsen att presentera kvitton på de stora lokalkostnader som
tillkommit under föregående verksamhetsår. Fredrik skall ta itu med detta efter
mötet.

\subpunkt{Presentation och diskussion kring revisionsberättelse för styrelsen
  2018/2019}
Edvin Åkerfeldt tar ordet och presenterar den revisionsberättelse han skickat
in för mötet. Se bilaga.


Sammanfattat anser Edvin att styrelsen 2018/2019 har skött sitt jobb, trots att
styrelsen haft färre protokollförda möten än tidigare samt att dessa varit
sena, rekommenderar han ansvarsfrihet då dessa är färdigrättade och uppladdade. 


\subpunkt{Beslut angående ansvarsfrihet för styrelsen 2018/2019}
Diskussion kring om föregående styrelse skulle beviljas ansvarsfrihet eller om
detta ska skjutas upp tills nästa föreningsmöte, så att styrelsen kan 
presentera kvitton för de stora utgifter för ombyggnationer som funnits, hålls. 

Louise lämnar in ett yrkande. Se bilaga.

\begin{beslut}
  \att välja ansvarsfrihet med tillägg, enligt yrkandet, inför nästa beslut.
\end{beslut}

\begin{beslut}
  \att välja ansvarsfrihet med tillägg.
\end{beslut}

\punkt{Behandling av motioner och propositioner}
\subpunkt{Preposition ang. Införande av kaffeavgift}
Prepositionen presenteras och diskuteras. Se bilaga. 


Mötet tar beslut om denna preposition med ett tillägg om specifik kostnad för
punkt 3. Se bilaga?

\begin{beslut}
  \att styrelsen ska kunna införa RFID autentisering vid användandet av
  kaffemaskinen genom LTU-kort.
\end{beslut}

\begin{beslut}
  \att införa en kostnad för kaffe och choklad.
\end{beslut}
%Sluten votering begärs för sista punktens specifika kostnad. 
%Alternativ 1 och 6, 0.10 och 0.69kr respektive får lika många röster.
%Efter detta hålls handuppräckning mellan dessa alternativ och 1 vinner. 

\begin{beslut}
  \att fastslå kostnaden till 0.10 kronor per kopp.
\end{beslut}

\punkt{Övriga frågor}
\subpunkt{Lång downtime}
Aran Nur frågar styrelsen och de rootmedlemmar på mötet varför
ThinLinc-terminalerna fortfarande är nere. root förklarar att det är en process
vilken är delvis avklarad. 

\subpunkt{Tacophest}
Mikael undrar när Tacophesten hålls, Anton berättar att datumet har bestämts
till 8:e November men att marknadsföring inte har börjat än. 

\subpunkt{LUDDLan/HöstLan}
Mikael frågar de medlemmar på mötet ifall det finns intresse att hålla ett
småskaligt höstLan, då hösten börjar dra sig mot ett slut, i T1.  

Visst intresse finns, och styrelsen uttrycker sig positivt till att medlemmar
skulle organisera tillställningar i lokalen. 

\end{document}
