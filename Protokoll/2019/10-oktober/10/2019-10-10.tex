\documentclass{protokoll}

\usepackage[utf8]{inputenc}

\bild{logo.pdf}     % Ändra om man vill ha en annan bild på framsidan.
\plats{T1}          % Plats som mötet hålls i
\typ{Medlemsmöte}  %Typ av möte: Styrelsemöte|Årsmöte|medlemsmöte etc.
\date{2019-10-07}  %Datum, form YYYY-MM-DD
\tid{18:00}{19:XX}   %Tid: {START}{SLUT}.
\organ{Styrelsen}   %Det organ som har hand om mötet: Styrelsen|Mötet|Medlemmarna
\organisation{Luleå Universitets Datorförening}
\forordf{Anton Johansson} 	            % Föreningsordförande
\ordf{Anton Johansson}%[titel] 	    % Mötesordförande  | (titel)
\sekr{Jens Lindholm}%[titel] 		% Mötessekreterare | (titel)
\justA{}{titel} 		% Justerare A      | titel
\justB{namn}{titel} 		% Justerare B      | titel

% Lista över övriga personer som närvarade vid mötet.
% Om det anses för många vid ett möte skriv: \person{<ANTAL> övriga medlemmar.}{}
\person{Josef Utbult}{Ledamot}
\person{Oscar Brink}{Kassör}
\person{Samuel Gradén}{Ledamot}
\person{Louise Sehlstedt}{Vice Ordförande}
\person{}{}
\person{}{}
\person{}{}
\person{}{}
\person{}{}
\person{}{}
\person{X andra}{}

%\adjung{} % Adjungerad person
%\adjung{} % Nästa adjungerade person
%\behorighet{Trump} % Om mötet är obehörigt, inkludera denna rad med anledning(-arna).

\begin{document}
% Punkter som hanteras automagiskt:
% Mötets högtidliga öppnande - när mötet öppnats
% Formalia
%       Val av mötesordförande - den som valdes tills mötesordförande
%       Val av mötessekreterare - den som valdes till mötessekreterare (alltså du)
%       Val avtvå justeringspersoner tillika rösträknare - de som valdes till protokolljusterare
%       Mötets behöriga utlysande - om mötet är behörigt är denna automagisk annars använd \behorighet
%       (Eventuella adjungeringar) - en samling av de som inadjungerats; kontrolleras via \adjung kommadot
% Mötets högtidliga avslutande - när mötet avslutats
\newpage  


\punkt{Godkännande av dagordning}
% Skriv eventuella ändringar i dagordningen och ändra beslutet.
\begin{beslut}
     \att fastslå dagordningen som skickats ut.
\end{beslut}


\punkt{Rapport från styrelsen}
\subpunkt{Presentation av verksamhetsplan för verksamhetsåret 2019/2020}
\subpunkt{Presentation av nya valda funktionärer av styrelsen sedan föregåendeföreningsmöte
\punkt{Rapport från revisor}
\punkt{Rapport från styrelsen 2018/2019}
\subpunkt{Presentation och diskussion kring föregående styrelsensverksamhetsberättelse}

\subpunkt{Presentation och diskussion kring föregående styrelsens
  kassaberättelse}

\subpunkt{Presentation och diskussion kring revisionsberättelse för styrelsen
  2018/2019}

\subpunkt{Beslut angående ansvarsfrihet för styrelsen 2018/2019}

\punkt{Behandling av motioner och propositioner}

\punkt{Övriga frågor}


\end{document}
