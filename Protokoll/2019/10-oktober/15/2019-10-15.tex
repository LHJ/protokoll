\documentclass{protokoll}

\usepackage[utf8]{inputenc}

\bild{logo.pdf}     % Ändra om man vill ha en annan bild på framsidan.
\plats{A2019}          % Plats som mötet hålls i
\typ{Styrelsemöte}  %Typ av möte: Styrelsemöte|Årsmöte|medlemsmöte etc.
\date{2019-10-15}  %Datum, form YYYY-MM-DD
\tid{16:42}{19:02}   %Tid: {START}{SLUT}.
\organ{Styrelsen}   %Det organ som har hand om mötet: Styrelsen|Mötet|Medlemmarna
\organisation{Luleå Universitets Datorförening}
\forordf{Anton Johansson} 	            % Föreningsordförande
\ordf{Anton Johansson}%[titel] 	    % Mötesordförande  | (titel)
\sekr{Jens Lindholm}%[titel] 		% Mötessekreterare | (titel)
\justA{Samuel Gradén}{Ledamot} 		% Justerare A      | titel
\justB{Louise Sehlstedt}{Vice Ordförande} 		% Justerare B      | titel

% Lista över övriga personer som närvarade vid mötet.
% Om det anses för många vid ett möte skriv: \person{<ANTAL> övriga medlemmar.}{}
\person{Oscar Brink}{Kassör}
\person{Josef Utbult}{Ledamot}

%\adjung{} % Adjungerad person
%\adjung{} % Nästa adjungerade person
%\behorighet{Trump} % Om mötet är obehörigt, inkludera denna rad med anledning(-arna).

\begin{document}
% Punkter som hanteras automagiskt:
% Mötets högtidliga öppnande - när mötet öppnats
% Formalia
%       Val av mötesordförande - den som valdes tills mötesordförande
%       Val av mötessekreterare - den som valdes till mötessekreterare (alltså du)
%       Val avtvå justeringspersoner tillika rösträknare - de som valdes till protokolljusterare
%       Mötets behöriga utlysande - om mötet är behörigt är denna automagisk annars använd \behorighet
%       (Eventuella adjungeringar) - en samling av de som inadjungerats; kontrolleras via \adjung kommadot
% Mötets högtidliga avslutande - när mötet avslutats
\newpage  


\punkt{Godkännande av dagordning}
% Skriv eventuella ändringar i dagordningen och ändra beslutet.
\begin{beslut}
     \att fastslå dagordningen som skickats ut.
\end{beslut}


\punkt{Bordlagda ärenden}
Inga bordlagda ärenden.


\punkt{Uppdatering av pågående projekt.}
\subpunkt{Tacophaest.}
Fakturaaddress behöver skickas in för att kunna betala för vakten. 


Eftersom Tacophesten tidigare har varit ett populärt event för medlemmar, bland
dessa medlemmar som har gått ut, rekommenderar styrelsen att framtida styrelser
håller Tacophester i lokaler utanför LTU så att detta inte blir ett problem. 


Bokning av matvaror från Snabbgross har ordnats.


Pris är satt till 75kr för medlemmar, 150kr för icke-medlemmar.
\subpunkt{LUDD-hack.}
Diskussion kring vad LUDD skulle stå för för uppgifter under ett framtida
LUDDhack hålls. 


Nuvarande plan för LUDD:s del under hacket är att hålla i någon form av kurs
eller kurser, backseat gaming med Shenzhen I/O samt en Pixelflutserver.


Utveckling av detta förväntas ske under teambuildingen med XP-el lördagen som
följer.
\subpunkt{Git-kurs vår 2020.}
Git-kursen har hållits en gång detta läsår och bedömdes gå väl. Mötet bedömer
att det skulle vara fördelaktigt med en labbdel, diskussion kring grundkoncept 
och ansvar förs. 


Styrelsen söker aktivt efter ansvariga för labbdelen för Git-kursen.
\subpunkt{Marknadsföring.}
Josef är marknadsföringsansvarig.

 
\punkt{Förstärkning av per capsulam beslut om Valberedning.}

\begin{beslut}
  \att förstärka beslutet taget den 7e oktober om att tillsätta Jonas Jacobsson
  samt Lisa Jonsson till valberedning respektive valberedning och tillkallande.
\end{beslut}

\punkt{Utvärdering av event.}
\subpunkt{Medlemsmöte.}
Styrelsen anser att medlemsmötet, vilket hölls 10:e oktober, sköttes väl och
gick relativt fort.


Våfflor tillagades utomhus vid cykelparkeringen, vilket fungerade väl, eftersom
LUDD har fått information att tillagning av osande mat inte är tillåtet inom
LUDD:s lokaler i framtiden.

\subpunkt{NCPC.}
NCPC hölls 5:e oktober, bristfällig information från NCPC för hur tävlingen ska 
organiseras på plats, innebar att LUDD inte kunde komma åt frågorna, komma åt
anmälning av lag i tid samt att frågor till NCPC inte besvarades i tid. 


Trots detta gick själva eventet relativt väl, anser ansvariga.
\punkt{Uppdatering angående uthyrningen av ljusställning till 6m för
grottphaesten.}
Grottphaesten vart inställd, och 6m kunde därför inte nyttja den ljusställning
som LUDD hade hyrt ut till 6m. Ljusställningen är återlämnad.


\begin{beslut}
  \att inte ta betalt för uthyrning av ljusställningen.
\end{beslut}

\punkt{Diskussion gällande bordugn.}
LUDD:s bordsugn har varit trasig sedan ett tag tillbaka. Efter lite
experimentering har det bedömts att ugnen bör kunna fixas med en ny timer,
vilken kompatibel motsvarande kan köpas från Aliexpress. 


Styrelsen kommer försöka fixa ugnen.
\punkt{Diskussion gällande inkommen motion.}
En motion har skickats in gällande ett beslut taget på LUDD:s senaste
medlemsmöte. Axel Alvarsson yrkar på att beslutet, vilket innebär att en kostnad
på LUDD:s maskinkaffe ska införas, ska ''makuleras''. 


Denna motion kommer tas upp på nästa medlemsmöte, förutsatt att medlemmen
fortfarande har ett intresse för att driva denna punkt vidare.
\punkt{Diskussion gällande trivseln i LUDD:s lokaler.}
Vissa medlemmar anses av styrelsen ha bristande hygien, vilken stör andra 
medlemmar.


En medlem uttryckt sig på ett sätt som har uppfattats till en annan medlem
online efter trakasserier i LUDD:s lokaler. Mötet diskuterar hur detta ska 
hanteras och ifall några konkreta regler för exakt hur långt en medlem får gå 
innan denne stängs av.


Ett informationsutskick till medlemmar om hur medlemmar ska gå till väga för
att anmäla problem de har med LUDD:s medlemmar eller lokaler, samt hur
styrelsen ska hantera dessa, ska gå ut.
\punkt{Övriga frågor}
% Icke beslutsfattande frågor och funderingar kring föreningen.
\subpunkt{Ventilation}
Efter diskussion med Irene Lernstål Askenryd har LUDD fått höra att det finns viss
möjlighet att få förlängd tid för ventilation, eventuellt i utbyte mot minskad
elförbrukning. 

\end{document}
