\documentclass{protokoll}

\usepackage[utf8]{inputenc}

\bild{logo.pdf}     % Ändra om man vill ha en annan bild på framsidan.
\plats{T1}          % Plats som mötet hålls i
\typ{Styrelsemöte}  %Typ av möte: Styrelsemöte|Årsmöte|medlemsmöte etc.
\date{2019-10-15}  %Datum, form YYYY-MM-DD
\tid{START}{SLUT}   %Tid: {START}{SLUT}.
\organ{Styrelsen}   %Det organ som har hand om mötet: Styrelsen|Mötet|Medlemmarna
\organisation{Luleå Universitets Datorförening}
\forordf{Anton Johansson} 	            % Föreningsordförande
\ordf{Anton Johansson}%[titel] 	    % Mötesordförande  | (titel)
\sekr{Jens Lindholm}%[titel] 		% Mötessekreterare | (titel)
\justA{namn}{titel} 		% Justerare A      | titel
\justB{namn}{titel} 		% Justerare B      | titel

% Lista över övriga personer som närvarade vid mötet.
% Om det anses för många vid ett möte skriv: \person{<ANTAL> övriga medlemmar.}{}
\person{namn}{titel}
\person{}{}

%\adjung{} % Adjungerad person
%\adjung{} % Nästa adjungerade person
%\behorighet{Trump} % Om mötet är obehörigt, inkludera denna rad med anledning(-arna).

\begin{document}
% Punkter som hanteras automagiskt:
% Mötets högtidliga öppnande - när mötet öppnats
% Formalia
%       Val av mötesordförande - den som valdes tills mötesordförande
%       Val av mötessekreterare - den som valdes till mötessekreterare (alltså du)
%       Val avtvå justeringspersoner tillika rösträknare - de som valdes till protokolljusterare
%       Mötets behöriga utlysande - om mötet är behörigt är denna automagisk annars använd \behorighet
%       (Eventuella adjungeringar) - en samling av de som inadjungerats; kontrolleras via \adjung kommadot
% Mötets högtidliga avslutande - när mötet avslutats
\newpage  


\punkt{Godkännande av dagordning}
% Skriv eventuella ändringar i dagordningen och ändra beslutet.
\begin{beslut}
     \att fastslå dagordningen som skickats ut.
\end{beslut}

\punkt{Uppdatering av pågående projekt.}
\subpunkt{Tacophaest.}

\subpunkt{LUDD-hack.}

\subpunkt{Git-kurs vår 2020.}

\subpunkt{Marknadsföring.}


\punkt{Förstärkning av per capsulam beslut om Valberedning.}

\punkt{Utvärdering av event.}
\subpunkt{Medlemsmöte.}

\subpunkt{NCPC.}


\punkt{Uppdatering angående uthyrningen av ljusställning till 6m för
grottphaesten.}


\punkt{Diskussion gällande bordugn.}


\punkt{Diskussion gällande inkommen motion.}


\punkt{Diskussion gällande trivseln i LUDD:s lokaler.}

\punkt{Övriga frågor}
% Icke beslutsfattande frågor och funderingar kring föreningen.
Inga övriga frågor.


\end{document}
