\documentclass{protokoll}

\usepackage[utf8]{inputenc}

\bild{logo.pdf}     % Ändra om man vill ha en annan bild på framsidan.
\plats{A2019}          % Plats som mötet hålls i
\typ{Styrelsemöte}  %Typ av möte: Styrelsemöte|Årsmöte|medlemsmöte etc.
\datum{2019-09-01}  %Datum, form YYYY-MM-DD
\tid{11:02}{13:23}   %Tid: {START}{SLUT}.
\organ{Styrelsen}   %Det organ som har hand om mötet: Styrelsen|Mötet|Medlemmarna
\organisation{Luleå Universitets Datorförening}
%\forordf{Anton Johansson} 	            % Föreningsordförande
\ordf{Anton Johansson}%[titel] 	    % Mötesordförande  | (titel)
\sekr{Jens Lindholm}%[titel] 		% Mötessekreterare | (titel)
\justA{Louise Sehlstedt}{Vice Ordförande} 		% Justerare A      | titel
\justB{Oscar Brink}{Kassör} 		% Justerare B      | titel

% Lista över övriga personer som närvarade vid mötet.
% Om det anses för många vid ett möte skriv: \person{<ANTAL> övriga medlemmar.}{}
\person{Josef Utbult}{Ledamot}
\person{Samuel Gradén}{Ledamot}

%\adjung{} % Adjungerad person
%\adjung{} % Nästa adjungerade person
%\behorighet{Trump} % Om mötet är obehörigt, inkludera denna rad med anledning(-arna).

\begin{document}
% Punkter som hanteras automagiskt:
% Mötets högtidliga öppnande - när mötet öppnats
% Formalia
%       Val av mötesordförande - den som valdes tills mötesordförande
%       Val av mötessekreterare - den som valdes till mötessekreterare (alltså du)
%       Val avtvå justeringspersoner tillika rösträknare - de som valdes till protokolljusterare
%       Mötets behöriga utlysande - om mötet är behörigt är denna automagisk annars använd \behorighet
%       (Eventuella adjungeringar) - en samling av de som inadjungerats; kontrolleras via \adjung kommadot
% Mötets högtidliga avslutande - när mötet avslutats
\newpage  


\punkt{Godkännande av dagordning}
% Skriv eventuella ändringar i dagordningen och ändra beslutet.
\begin{beslut}
     \att fastslå dagordningen som skickats ut.
\end{beslut}

\punkt{Bordlagda ärenden}
% Har det bordlagts ärenden från förra mötet?
Inga bordlagda ärenden.

\punkt{Utvärdera nolle-p}
Konsensus är att nolle-p har gått tillräckligt bra. De saker som anses fungerat
mindre bra har varit allmänt delande av information mellan styrelsemedlemmarna 
som skickats ut till enbart delar av styrelsen. 


18 nya medlemmar har registrerats sedan början av nolleperioden. 


Att färre rekryterades denna nolleperiod jämfört med tidigare år bedöms vara
till stor del på grund av att de förmåner som inte var datorrelaterade, kaffe,
utskrifter och studieytor, medvetet inte gjordes reklam för.
Även att LUDD inte medverkade på sektionsdagen samt att spelkvällen som hölls 
i var mindre och att LUDD inte tilläts göra aktiv reklam på denna tros ha
påverkat antalet nya medlemmar. 



\punkt{Införande av pris på LUDDs kaffe}
\subpunkt{Ta upp det på medlemsmöte}
Förra styrelsen höll diskussion att införa ett symboliskt pris för kaffe på 
LUDD, vilken skulle täcka delar av utgifterna för kaffe samt ge statistik över
konsumtion.


Eventuella införandet av detta planeras tas upp på nästa medlemsmöte. 

\punkt{Kurser under läsåret.}
\subpunkt{Git- och LaTeX-kurs}
Git-kursen har planerats in att hållas på Torsdag 12:e September, klockan 
18:00, av Anders ''ankan'' Engström. A1545 har bokats för kursen. 


LaTeX-kursen planeras hållas på Torsdag 19:e September, klockan 18:00, av Jens
''lh'' Lindholm. John ''curious'' Elfberg Larsson ska kontaktas och förfrågas
ifall denne kan tänkas hålla i labbdelen av kursen. A3019 planeras bokas för
kursen. 


Oscar Brink tar på sig ansvar att designa posters, samt skapa Facebookevent för
Git-kursen. 


Josef Utbult tar på sig att skapa ett Facebookevent för LaTeX-kursen.


Anton Johansson tar på sig att kontakta Teknologkåren för marknadsföring.
\subpunkt{C++ kurs}
Anton Johansson har skickat mail till Fredrik Bengtsson för att få förslag på
föreläsare men inte fått svar, Anton tar på sig att kontakta Fredrik i person.


Oscar Brink tar på sig att kontakta Per Lindgren för att få förslag på
föreläsare. 

\punkt{Valberedningen}
LUDD står utan valberedning, styrelsen har fått ansvaret att tillsätta två
valberedande. 


Lisa Johansson har uttryckt att hon kan tänka sig vara del av valberedningen
även detta år, beroende på vem som också skulle vara med i valberedningen. 

\punkt{Återställning av protokoll + docs}
Efter intrånget och utplånandet av LUDD:s GitLab krashade storageklustret under
en uppdatering, vilken tog bort alla förändringar som hade gjorts efter
återställningen. Alla protokoll som inte är utskrivna och påskrivna finns som
källkod, för LUDD:s samlade styrdokument finns de senaste förändringarna i två
olika lokala repon som ska merge:as ihop. 


Jens Lindholm tar på sig ansvaret för att dessa ska återställas till så nära
som går deras ursprungliga status. 

\punkt{Kolla vad som ska utföras inför/under medlemsmötet så vi är väl 
förberedda}
Louise Sehlstedt presenterar och förklarar vad som måste göras under
medlemsmötet som ska hållas under läsperiod 1. 

\begin{beslut}
  \att det första medlemsmötet ska hållas torsdag den 10:e Oktober klockan
  18:00.
\end{beslut}
\punkt{Planering och marknadsföring för torsdagshack}
Passande nivå på reklam för Torsdagshack diskuteras och föreslås vara posters 
på LUDD och antingen återkommande events eller regelbundna inlägg på väggen. 


Skriva ut posters för både LUDD:s lokal och omkring LTU under tidiga läsperiod
1, samt regelbundna inlägg på LUDD:s vägg. 
\punkt{Mål och visioner}
\subpunkt{Vad är mål och visionerna}
Louise presenterar LUDD:s Mål och Visioner samt den nuvarande verksamhetsplanen.  

\subpunkt{Hur ska vi göra för att uppnå mål och visionerna}
Mötet har tagit fram en preliminär verksamhetsplan som kommer förmedlas till
medlemmarna under mötet, men kommer arbetas med från och med detta möte.

\punkt{Övriga frågor}
% Icke beslutsfattande frågor och funderingar kring föreningen.
Inga övriga frågor.

\end{document}
