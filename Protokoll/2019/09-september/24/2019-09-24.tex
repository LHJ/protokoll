\documentclass{protokoll}

\usepackage[utf8]{inputenc}

\bild{logo.pdf}     % Ändra om man vill ha en annan bild på framsidan.
\plats{T1}          % Plats som mötet hålls i
\typ{Styrelsemöte}  %Typ av möte: Styrelsemöte|Årsmöte|medlemsmöte etc.
\date{2019-09-24}  %Datum, form YYYY-MM-DD
\tid{17:59}{2X:00}   %Tid: {START}{SLUT}.
\organ{Styrelsen}   %Det organ som har hand om mötet: Styrelsen|Mötet|Medlemmarna
\organisation{Luleå Universitets Datorförening}
\forordf{Anton Johansson} 	            % Föreningsordförande
\ordf{Anton Johansson}%[titel] 	    % Mötesordförande  | (titel)
\sekr{Jens Lindholm}%[titel] 		% Mötessekreterare | (titel)
\justA{Louise Sehlstedt}{Vice ordförande} 		% Justerare A      | titel
\justB{Oscar Brink}{Kassör} 		% Justerare B      | titel

% Lista över övriga personer som närvarade vid mötet.
% Om det anses för många vid ett möte skriv: \person{<ANTAL> övriga medlemmar.}{}
\person{Josef Utbult}{Ledamot}

%\adjung{} % Adjungerad person
%\adjung{} % Nästa adjungerade person
%\behorighet{Trump} % Om mötet är obehörigt, inkludera denna rad med anledning(-arna).

\begin{document}
% Punkter som hanteras automagiskt:
% Mötets högtidliga öppnande - när mötet öppnats
% Formalia
%       Val av mötesordförande - den som valdes tills mötesordförande
%       Val av mötessekreterare - den som valdes till mötessekreterare (alltså du)
%       Val avtvå justeringspersoner tillika rösträknare - de som valdes till protokolljusterare
%       Mötets behöriga utlysande - om mötet är behörigt är denna automagisk annars använd \behorighet
%       (Eventuella adjungeringar) - en samling av de som inadjungerats; kontrolleras via \adjung kommadot
% Mötets högtidliga avslutande - när mötet avslutats
\newpage  


\punkt{Godkännande av dagordning}
% Skriv eventuella ändringar i dagordningen och ändra beslutet.
\begin{beslut}
     \att fastslå dagordningen som skickats ut.
\end{beslut}

\punkt{Bordlagda ärenden}
Inga bordlagda ärenden.
\punkt{Diskussion gällande eventuellt samarbete med Academic Work}
Efter diskussion inom styrelsen och med medlemmar har styrelsen kommit fram
till att, i skrivande stund, inte söka ett samarbete med Academic Work. 


Academic Work har kontaktats och meddelats om detta.

\punkt{Diskussion gällande NCPC 2019 (Nordic Collegiate Programming Contest)}
Edvin Åkerfeldt har skrivit upp LUDD som hub för de som vill delta under NCPC 
2019, vilken kommer att hållas den 5:e Oktober. 


Anton tar på sig ansvaret att kontakta SRT för sponsring, boka sal samt för 
NCPC som helhet.


Oscar tar på sig ansvaret att skriva ut posters. 

Josef tar på sig ansvaret att skapa facebookevent för NCPC.
\punkt{Utvärdering av LaTeX-kursen}
5 personer medverkade under kursen, 

Saker som föreslås kunna förbättras till nästa gång inkom, bland andra: 


Hålla tidigare, för att få dataettor innan inlämning för D0015E, till exempel 
andra veckan i läsperiod 1. Alternativt hålla den senare, under sena läsperiod
2.  


Få ut digital marknadsföring i tid. 


Mer stöd från resten av styrelsen för de som håller i den, med mer assistenter
under labbdelen. 
\punkt{Medlemsmöte}
 \subpunkt{Presentation och komplettering av hjälpmedel}
Louise har förberett mallar och hjälpmaterial för förenkla föreningsmöten vilka 
ligger på styrelsens Google Drive. 
Louise sammanfattar några av dessa. 
\subpunkt{Utformande av kallelse}

\subpunkt{Förslag om datum och tid för övning}

\punkt{Tacophesten}
 \subpunkt{Förslag om ansvariga}

\subpunkt{Förslag om datum}

\subpunkt{Diskussion gällande Burn-såsen}

\punkt{Information gällande styrelsens teambuilding med XP-el:s styrelse}

\punkt{Beslut att godkänna uthyrning av ljusställning för 1000 SEK till 6m för
grottphaesten}

\punkt{Diskussion gällande att styrelsen ska utse en marknadsföringsansvarig}

\punkt{Övriga ärenden}

\end{document}
