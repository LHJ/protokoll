\documentclass{protokoll}

\usepackage[utf8]{inputenc}

\bild{logo.pdf}     % Ändra om man vill ha en annan bild på framsidan.
\plats{T1}          % Plats som mötet hålls i
\typ{Styrelsemöte}  %Typ av möte: Styrelsemöte|Årsmöte|medlemsmöte etc.
\datum{2019-09-24}  %Datum, form YYYY-MM-DD
\tid{17:59}{20:10}   %Tid: {START}{SLUT}.
\organ{Styrelsen}   %Det organ som har hand om mötet: Styrelsen|Mötet|Medlemmarna
\organisation{Luleå Universitets Datorförening}
\forordf{Anton Johansson} 	            % Föreningsordförande
\ordf{Anton Johansson}%[titel] 	    % Mötesordförande  | (titel)
\sekr{Jens Lindholm}%[titel] 		% Mötessekreterare | (titel)
\justA{Louise Sehlstedt}{Vice ordförande} 		% Justerare A      | titel
\justB{Oscar Brink}{Kassör} 		% Justerare B      | titel

% Lista över övriga personer som närvarade vid mötet.
% Om det anses för många vid ett möte skriv: \person{<ANTAL> övriga medlemmar.}{}
\person{Josef Utbult}{Ledamot}

%\adjung{} % Adjungerad person
%\adjung{} % Nästa adjungerade person
%\behorighet{Trump} % Om mötet är obehörigt, inkludera denna rad med anledning(-arna).

\begin{document}
% Punkter som hanteras automagiskt:
% Mötets högtidliga öppnande - när mötet öppnats
% Formalia
%       Val av mötesordförande - den som valdes tills mötesordförande
%       Val av mötessekreterare - den som valdes till mötessekreterare (alltså du)
%       Val avtvå justeringspersoner tillika rösträknare - de som valdes till protokolljusterare
%       Mötets behöriga utlysande - om mötet är behörigt är denna automagisk annars använd \behorighet
%       (Eventuella adjungeringar) - en samling av de som inadjungerats; kontrolleras via \adjung kommadot
% Mötets högtidliga avslutande - när mötet avslutats
\newpage  


\punkt{Godkännande av dagordning}
% Skriv eventuella ändringar i dagordningen och ändra beslutet.
\begin{beslut}
     \att fastslå dagordningen som skickats ut.
\end{beslut}

\punkt{Bordlagda ärenden}
Inga bordlagda ärenden.
\punkt{Diskussion gällande eventuellt samarbete med Academic Work}
Efter diskussion inom styrelsen och med medlemmar har styrelsen kommit fram
till att, i skrivande stund, inte söka ett samarbete med Academic Work. 


Academic Work har kontaktats och meddelats om detta.

\punkt{Diskussion gällande NCPC 2019 (Nordic Collegiate Programming Contest)}
Edvin Åkerfeldt har skrivit upp LUDD som hub för de som vill delta under NCPC 
2019, vilken kommer att hållas den 5:e Oktober. 


Anton tar på sig ansvaret att kontakta SRT för sponsring, boka sal samt för 
NCPC som helhet.


Oscar tar på sig ansvaret att skriva ut posters. 

Josef tar på sig ansvaret att skapa facebookevent för NCPC.
\punkt{Utvärdering av LaTeX-kursen}
5 personer medverkade under kursen, det låga antalet medverkande bedömdes bero
  på, bland annat, bristfällig marknadsföring inför den.
Marknadsföringen led på grund av hur tätt kursen hölls inpå Git-kursen.


Saker som föreslås kunna förbättras till nästa gång inkom, bland andra: 


Hålla tidigare, för att få dataettor innan inlämning för D0015E, till exempel 
andra veckan i läsperiod 1. Alternativt hålla den senare, under sena läsperiod
2.  


Få ut digital marknadsföring i tid. 


Mer stöd från resten av styrelsen för de som håller i den, med fler assistenter
under labbdelen. 
\punkt{Medlemsmöte}
 \subpunkt{Presentation och komplettering av hjälpmedel}
Louise har förberett mallar och hjälpmaterial för förenkla föreningsmöten vilka 
ligger på styrelsens Google Drive. 

Louise sammanfattar några av dessa. 

\subpunkt{Utformande av kallelse}
Louise presenterar ett utdrag för mötets kallelse.

\subpunkt{Förslag om datum och tid för övning}
Mötet diskuterar när det planerade övningsmötet ska hållas.


Föreslaget datum och tid är klockan 18:00-$\infty$ den 7:e oktober.
\punkt{Tacophesten}
\subpunkt{Förslag om ansvariga}
Josef Utbult och Samuel Gradén, speciellt Samuel, är ansvariga för utförande av 
Tacophesten. 

Anton tar på sig ansvar för bokning av A-torget, kontakt med Stefan Carlsson 
samt bokning av vakt.

Josef tar på sig ansvar för att ta kontakt med godtyckligt sexmästeri för 
servering av alkohol samt STUK för bokning av värmeskåp.

Oscar är ansvarig för att kontakta Snabbgross för bokning av köttfärs.
\subpunkt{Förslag om datum}
Tacophesten bestäms hållas den 8:e november med preliminär tid för insläpp
klockan 18:00.
\subpunkt{Diskussion gällande Burn-såsen}
Jonas ''blondie'' Sundén och Dan ''wimbledon'' Wimble ska kontaktas av Josef 
gällande tillagning av Burn. 

\punkt{Information gällande styrelsens teambuilding med XP-el:s styrelse}
Louise har, tillsammans med en kontaktperson från XP-el:s styrelse, planerat en
gemensam teambuildingdag för de två styrelserna. Tid och datum är bestämt 15:00
19:e oktober. 

\punkt{Beslut att godkänna uthyrning av ljusställning för 1000 SEK till 6m för
grottphaesten}
\begin{beslut}
  \att godkänna uthyrningen av LUDD:s ljusställning till 6m under helgen
  5:e-7:e oktober för 1000kr.
\end{beslut}

\punkt{Diskussion gällande att styrelsen ska utse en marknadsföringsansvarig}
Josef har, sedan tidigare, tagit på sig ansvaret för LUDD:s sociala medier.  

Diskussion kring huruvida Josef är villig att ta på sig ett mer komplett
ansvar för generell marknadsföring för LUDD hålls.
\begin{beslut}
  \att utse Josef Utbult till marknadsföringsansvarig för LUDD.
\end{beslut}

\punkt{Övriga ärenden}
\subpunkt{SRT:s artikel om LUDD}
SRT skriver en artikel om LUDD, Johan Jatko har informerat styrelsen att SRT
önskar ha nyare bilder av LUDD och LUDD:s styrelse, samt gärna gamla LUDDare. 

Mötet föreslår måndag 30:e september eller tisdag 1:a oktober till SRT.

\subpunkt{Merchandise för LUDD}
Josef har kollat på webbkamera-skydd som skulle kunna köpas in med egna teman. 


Tygmärken och hoodies diskuteras också. 

\subpunkt{Rockmålning}
Oscar framför att styrelsen fortfarande inte har målat sina rockar för året. 


Efter diskussion föreslås det att rockarna ska målas måndag 30:e september, 
klockan 14:45.
\end{document}
