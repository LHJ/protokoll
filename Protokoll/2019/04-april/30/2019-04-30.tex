\documentclass{protokoll}

\usepackage[utf8]{inputenc}

\bild{logo.pdf}     % Ändra om man vill ha en annan bild på framsidan.
\plats{LUDD:s Discord}          % Plats som mötet hålls i
\typ{Styrelsemöte}  %Typ av möte: Styrelsemöte|Årsmöte|medlemsmöte etc.
\datum{2019-04-30}  %Datum, form YYYY-MM-DD
\tid{15:06}{15:10}   %Tid: {START}{SLUT}.
\organ{Styrelsen}   %Det organ som har hand om mötet: Styrelsen|Mötet|Medlemmarna
\organisation{Luleå Universitets Datorförening}
\forordf{August Eriksson} 	            % Föreningsordförande
\ordf{Anton Johansson}[Ledamot]%[titel] 	    % Mötesordförande  | (titel)
\sekr{Jens Lindholm}%[titel] 		% Mötessekreterare | (titel)
\justA{Edvin Åkerström}{Ledamot} 		% Justerare A      | titel
\justB{Fredrik Pettersson}{Kassör} 		% Justerare B      | titel

% Lista över övriga personer som närvarade vid mötet.
% Om det anses för många vid ett möte skriv: \person{<ANTAL> övriga medlemmar.}{}
%\person{Fredrik Pettersson}{Kassör}
\person{Emil Kitti}{Huvudsystemansvarig}
%\person{Anton Johansson}{Ledamot}
%\person{Edvin Åkerström}{Ledamot}


%\adjung{} % Adjungerad person
%\adjung{} % Nästa adjungerade person
%\behorighet{Trump} % Om mötet är obehörigt, inkludera denna rad med anledning(-arna).

\begin{document}
% Punkter som hanteras automagiskt:
% Mötets högtidliga öppnande - när mötet öppnats
% Formalia
%       Val av mötesordförande - den som valdes tills mötesordförande
%       Val av mötessekreterare - den som valdes till mötessekreterare (alltså du)
%       Val avtvå justeringspersoner tillika rösträknare - de som valdes till protokolljusterare
%       Mötets behöriga utlysande - om mötet är behörigt är denna automagisk annars använd \behorighet
%       (Eventuella adjungeringar) - en samling av de som inadjungerats; kontrolleras via \adjung kommadot
% Mötets högtidliga avslutande - när mötet avslutats
\newpage  


\punkt{Godkännande av dagordning}
% Skriv eventuella ändringar i dagordningen och ändra beslutet.
\begin{beslut}
     \att fastslå dagordningen som skickats ut.
\end{beslut}

\punkt{Bordlagda ärenden}
% Har det bordlagts ärenden från förra mötet?
Inga bordlagda ärenden.


\punkt{Beslut kring roots motion att avsätta 16 000 000 boschmoney till
elinstallation, motsvarande 16 000 svenska kronor}
Elen i serverhallen ska dras om, motivaterat delvis med att det skulle minska 
risk för bränder och få bättre koll på strömförbrukningen. 
\begin{beslut}
  \att avlägga 16 000 000 boschmoney, motsvarande 16 000kr, från budgetposten
  ''Lokal'' för att göra en elinstallation i serverhallen.
\end{beslut}

\punkt{Övriga ärenden}
% Icke beslutsfattande frågor och funderingar kring föreningen.
Inga övriga ärenden.


\end{document}
