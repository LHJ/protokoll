\documentclass{protokoll}

\usepackage[utf8]{inputenc}

\bild{logo.pdf}     % Ändra om man vill ha en annan bild på framsidan.
\plats{T1}          % Plats som mötet hålls i
\typ{Styrelsemöte}  %Typ av möte: Styrelsemöte|Årsmöte|medlemsmöte etc.
\datum{2019-04-24}  %Datum, form YYYY-MM-DD
\tid{15:00}{16:03}   %Tid: {START}{SLUT}.
\organ{Styrelsen}   %Det organ som har hand om mötet: Styrelsen|Mötet|Medlemmarna
\organisation{Luleå Universitets Datorförening}
\forordf{August Eriksson} 	            % Föreningsordförande
\ordf{Anton Johansson}[Ledamot]%[titel] 	    % Mötesordförande  | (titel)
\sekr{Jens Lindholm}%[titel] 		% Mötessekreterare | (titel)
\justA{Josef Utbult}{Medlem} 		% Justerare A      | titel
\justB{Samuel Gradén}{Medlem} 		% Justerare B      | titel

% Lista över övriga personer som närvarade vid mötet.
% Om det anses för många vid ett möte skriv: \person{<ANTAL> övriga medlemmar.}{}
%\person{Emil Kitti}{Huvudsystemansvarig}
%\person{Anton Johansson}{Ledamot}
\person{Edvin Åkerström}{Ledamot}
%\person{Louise Sehlstedt}{Medlem}
%\person{John Elfberg Larsson}{Medlem}
\person{Oscar Brink}{Medlem}
%\person{Samuel Gradén}{Medlem}
%\person{Josef Utbult}{Medlem}

%\adjung{} % Adjungerad person
%\adjung{} % Nästa adjungerade person
%\behorighet{Trump} % Om mötet är obehörigt, inkludera denna rad med anledning(-arna).

\begin{document}
% Punkter som hanteras automagiskt:
% Mötets högtidliga öppnande - när mötet öppnats
% Formalia
%       Val av mötesordförande - den som valdes tills mötesordförande
%       Val av mötessekreterare - den som valdes till mötessekreterare (alltså du)
%       Val avtvå justeringspersoner tillika rösträknare - de som valdes till protokolljusterare
%       Mötets behöriga utlysande - om mötet är behörigt är denna automagisk annars använd \behorighet
%       (Eventuella adjungeringar) - en samling av de som inadjungerats; kontrolleras via \adjung kommadot
% Mötets högtidliga avslutande - när mötet avslutats
\newpage  


\punkt{Godkännande av dagordning}
% Skriv eventuella ändringar i dagordningen och ändra beslutet.
\begin{beslut}
     \att fastslå dagordningen som skickats ut med tillägget utav punkten 
     ''LUDD är hackat'' efter punkten ''Beslut huruvida CCLan ska hållas år
     2020.''
\end{beslut}

\punkt{Bordlagda ärenden}
% Har det bordlagts ärenden från förra mötet?
Inga bordlagda ärenden.

\punkt{Beslut huruvida CCLan ska hållas år 2020}
Större delen av blivande styrelsen är medverkande på mötet för att diskutera
ifall CCLan ska hållas under deras verksamhetsår. 

Att CCLan skulle ha stort värde relativt energin som måste läggas ner på det
anses inte vara fallet. 

Det föreslås att engagerade medlemmar får gärna arrangera LAN, men att det inte
ska vara styrelsens ansvar.

Det föreslås att styrelsen, ifall LUDD drar sig ur CCLan projektet, skulle
hjälpa TAFS i Skellefteå ifall de behöver teknisk assistans. 

\begin{beslut}
  \att på nya styrelsens vägnar, styrelsen inte ska vara ansvarig för att hålla
  ett CCLan verksamhetsåret 2019/2020.
\end{beslut}

\punkt{LUDD är hackat}
Mötet ber tillgängliga root i lokalen förklara vad som har hänt.  

I runda slängar har två root fått sina konton hackade i en riktad attack mot
LUDD:s git server, målet var ett annat projekt som en av rötterna arbetade på,
efter att inte ha kommit åt detta projekt tog angriparen bort alla projekt
som låg under ''ludd'' gruppen på GitLab. 

Detta innebär att repos för bland annat styrdokument och protokoll har tagits
bort. Vissa av dessa finns som lokala kopior, andra, till exempel LUDD:s issue
tracker, är borta.

Mötet uppmuntrar starkt rootgruppen att i framtiden ha faktiska backups
istället för enbart vissa projekt som de anser nödvändiga.

\punkt{Övriga ärenden}
% Icke beslutsfattande frågor och funderingar kring föreningen.
\subpunkt{C++ Kurs}
En motion har inkommit från en medlem där denna yrkar på att LUDD ska hålla
i en C++ kurs. 
Mötet diskuterar tillgång till någon som är villig att hålla i en nybörjarkurs
i C++ som också är kunnig inom språket. 

Det föreslås att kontakta SRT för att fråga ifall det finns någon professor som
skulle kunna föreläsa under en sådan kurs. 

Mötet ser positivt på möjligheten att hålla i en kurs 

\subpunkt{Git-kursen}
Det diskuteras ifall det är värt att hålla i en Git-kurs med tanke på hur kort 
tid det är kvar på läsåret. Det föreslås att den blivande styrelsen skulle 
hålla den under LP1 istället, då andraårs datastudenter introduceras till Git 
under den läsperioden. Första årets datastudenter kan också marknadsföras till
också då de kommer 

August tar på sig att kontakta Anders ''ankan'' Engström och fråga ifall han 
kan hålla i kursen under LP1.

Anton tar på sig att kontakta Per Lindgren för att se ifall LUDD kan få eller
ifall han skulle kunna markandsföra LUDD:s Git-kurs under momentet då Git 
ska introduceras under kursen D0013E som Per brukar hålla i.

\end{document}
