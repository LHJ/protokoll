\documentclass{protokoll}

\usepackage[utf8]{inputenc}

\bild{logo.pdf}     % Ändra om man vill ha en annan bild på framsidan.
\plats{T1}          % Plats som mötet hålls i
\typ{Styrelsemöte}  %Typ av möte: Styrelsemöte|Årsmöte|medlemsmöte etc.
\datum{2019-03-14}  %Datum, form YYYY-MM-DD
\tid{18:25}{21:54}   %Tid: {START}{SLUT}.
\organ{Styrelsen}   %Det organ som har hand om mötet: Styrelsen|Mötet|Medlemmarna
\organisation{Luleå Universitets Datorförening}
\forordf{August Eriksson} 	            % Föreningsordförande
\ordf{August Eriksson}%[titel] 	    % Mötesordförande  | (titel)
\sekr{Jens Lindholm}%[titel] 		% Mötessekreterare | (titel)
\justA{Mikael Hedkvist}{Medlem} 		% Justerare A      | titel
\justB{Viktor Sonesten}{Medlem} 		% Justerare B      | titel

% Lista över övriga personer som närvarade vid mötet.
% Om det anses för många vid ett möte skriv: \person{<ANTAL> övriga medlemmar.}{}
\person{Frida Mikkelä}{Vice Ordförande}
\person{Emil Kitti}{Huvudsystemansvarig}
\person{Fredrik Pettersson}{Kassör}
\person{Edvin Åkerfeldt}{Revisor}
\person{Edvin Åkerström}{Ledamot}
\person{Anton Johansson}{Ledamot}
\person{Lisa Johansson}{Valberedning}
\person{Oscar Brink}{Valberedning}
\person{John Elfberg Larsson}{Medlem}
\person{Aran Nur}{Medlem}
\person{Louise Sehlstedt}{Medlem}
\person{Patrick Hjelte}{Medlem}
\person{Erik Viklund}{Medlem}
\person{Rickard Åström}{Medlem}
\person{7 andra}{}
% Några fler LUDD gamlingar kom 20:30

%\adjung{} % Adjungerad person
%\adjung{} % Nästa adjungerade person
%\behorighet{Trump} % Om mötet är obehörigt, inkludera denna rad med anledning(-arna).

\begin{document}
% Punkter som hanteras automagiskt:
% Mötets högtidliga öppnande - när mötet öppnats
% Formalia
%       Val av mötesordförande - den som valdes tills mötesordförande
%       Val av mötessekreterare - den som valdes till mötessekreterare (alltså du)
%       Val avtvå justeringspersoner tillika rösträknare - de som valdes till protokolljusterare
%       Mötets behöriga utlysande - om mötet är behörigt är denna automagisk annars använd \behorighet
%       (Eventuella adjungeringar) - en samling av de som inadjungerats; kontrolleras via \adjung kommadot
% Mötets högtidliga avslutande - när mötet avslutats
\newpage  


\punkt{Godkännande av dagordning}
% Skriv eventuella ändringar i dagordningen och ändra beslutet.
\begin{beslut}
     \att fastslå dagordningen som skickats ut.
\end{beslut}

\punkt{Bordlagda ärenden}
\subpunkt{Ansvarsfrihet för styrelsen verksamhetsåret 2017/2018}
De medlemmar av föregående styrelse som var närvarande gav 

Mikael yrkar på att med den information som framförts bör styrelsen beviljas
den ansvarsfrihet som tidigare nekats. 

\begin{beslut}
  \att bevilja ansvarsfrihet för styrelsen 2017/2018
\end{beslut}

\punkt{Rapport från styrelsen}
Fredrik Pettersson presenterar en rapport för mötet, se bilaga.

\punkt{Rapport från revisor}
Revisor Edvin Åkerfeldt berättar för mötet att han anser att ekonomin verkar
skötas väl och att kassör gör sitt jobb. 

\punkt{Ekonomisk redovisning}
Fredrik presenterar LUDDs ekonomi för mötet. 
Lite diverse fel i formatteringen pekas ut och förklaras vara på grund av
hanteringsfel av Excel. 

Mikael yrkar på att lägga till punkten ''Fastställande av budget och medlemsavgift för nästkommande
verksamhetsår.'' under punkten ''Ekonomisk redovisning''. 
\begin{beslut}
\att lägga till punkten ''Fastställande av budget och medlemsavgift för nästkommande
verksamhetsår.'' under punkten ''Ekonomisk redovisning''.
\end{beslut}
\punkt{Fastställande av budget och medlemsavgift för nästkommande
verksamhetsår.}
Fredrik presenterar sitt budgetförslag för verksamhetsåret 2019/2020.

Förslag på att höja budgetposten ''Datortillbehör'' då den regelbundet
överstigs, diskussion kring huruvida om den förväntas överstigas och huruvida
detta skulle vara ett problem, då en budget är en riktlinje och inte något som
inte får brytas.  

Diskussion kring om någon annan post skulle sänkas för att kompensera för
eventuell ökning hålls. 

Louise och Emil Kitti yrkar på att öka budgetposten ''Datortillbehör'' till 
10000kr och att sänka ''Frakter och Transporter'' med 1000kr och ''Förbrukningsinventarie'' med 2000kr. 
\begin{beslut}
  \att jämka yrkandet. 
\end{beslut}

\punkt{Uppdatering av LUDDs mål och visioner}
August presenterar den föreslagna uppdatering av LUDD:s mål och visioner. 

Mikael yrkar på att ändra ordet ''stadigt'' på rad 258 i git-diff:en till ordet
''ökande''. 

Styrelsen jämkar sig med förslaget. 


Mikael yrkar på att behålla punkt 1.10., ''Världens geografiskt största LAN'',
som hade tagits bort i den föreslagna uppdateringen av mål och visioner. 

Styrelsen jämkar sig med förslaget. 

\begin{beslut}
  \att fastställa den föreslagna ändringen med de förändringar som styrelsen 
  jämkat sig med. 
\end{beslut}


\punkt{Förankring av tills vidare beslutade regler genom medlemsbeslut}
August presenterar den tillagda regeln 4.4.6. ''Samtalsklimat''.

\begin{beslut}
  \att förankra den tillagda regeln 4.4.6. ''Samtalsklimat.''
\end{beslut}

\punkt{Behandling motioner och propositioner}
Inga motioner eller propositioner har inkommit.

\punkt{Val av styrelse}
\subpunkt{Val av Ordförande}
Valberedningen nominerar Anton Johansson till Ordförande för verksamhetsåret 2019/2020.

Anton presenterar sig.

\begin{beslut}
  \att välja Anton Johansson till Ordförande för verksamhetsåret 2019/2020.
\end{beslut}

\subpunkt{Val av Vice Ordförande}
Valberedningen nominerar Louise Sehlstedt till Vice Ordförande för verksamhetsåret 2019/2020.

Louise presenterar sig.
\begin{beslut}
  \att välja Louise till Vice Ordförande för verksamhetsåret 2019/2020.
\end{beslut}

\subpunkt{Val av Kassör}
Samuel Gradén nominerar Oscar Brink till Kassör. 

Oscar presenterar sig själv.

\begin{beslut}
  \att välja Oscar till Kassör.
\end{beslut}

\subpunkt{Val av Sekreterare}
Valberedningen nominerar Jens Lindholm till Sekreterare för verksamhetsåret 2019/2020.

Jens Lindholm presenterar sig själv.

John Elfberg Larsson tog rollen som temporär mötessekreterare medan Jens blir
utfrågad och röstning sker.

% Jens is a cuckboi that has gone completely mental. Why do you want to be
% secretary again? Madness! Suck mah dick, you furry degenerate. - John
% Fuck John lmfao - lh

\begin{beslut}
  \att välja Jens Lindholm till Sekreterare för verksamhetsåret 2019/2020.
\end{beslut}

Jens Lindholm tog åter över rollen som mötessekreterare.

\subpunkt{Val av två Ledamöter}
Valberedningen nominerar Josef Utbult och Samuel Gradén till Ledamöter för 
verksamhetsåret 2019/2020.

Samuel och Josef presenterar sig själva.

\begin{beslut}
  \att välja Josef Utbult och Samuel Gradén till Ledamöter.
\end{beslut}

\subpunkt{Val av Huvudsystemansvarig}
Valberedningen och rootgruppen nominerar John Elberg Larsson till
Huvudsystemansvarig för verksamhetsåret 2019/2020.

\begin{beslut}
  \att välja John Elfberg Larsson till Huvudsystemansvarig för verksamhetsåret 
  2019/2020.
\end{beslut}

\subpunkt{Val av två personer till valberedning, varav en sammankallande}
Frida nominerar Aran Nur till valberedning samt sammankallande. 

Inga fler nomineringar godtas av de nominerade. %eller skriver man inga fler
%nomineringar inkommer?

Aran presenterar sig själv. 

\begin{beslut}
  \att lämna posten vakant. 
\end{beslut}

\begin{beslut}
  \att styrelsen åläggs tillsätta valberedning.
\end{beslut}

\punkt{Övriga val}
\subpunkt{Val av revisor och suppleant}
Valberedningen nominerar Edvin Åkerfeldt till revisor för verksamhetsåret 2019/2020.

Frida nominerar Fredrik till suppleant. 

Edvin Åkerfeldt nominerar August till revisor.

August nominerar Edvin Åkerfeldt till temporär mötesordförande. 

Mötet väljer Edvin Åkerfeldt till temporär mötesordförande medans August
utfrågas och röstning sker. 

\begin{beslut}
  \att välja August och Fredrik till revisor respektive suppleant för verksamhetsåret 2019/2020.
\end{beslut}

\punkt{Övriga frågor}
\subpunkt{El i lokalen}
August framför att root har planer för att ordna den elektriska situationen
i LUDDs lokal men att kapaciteten inte kommer att öka direkt.

\subpunkt{Skräp på bord}
Mikael framför att han anser att lokalen och dess bordsytor är nerstökade med 
diverse teknologiskt skräp i stil av sladdar och kretskort. Han föreslår 
hyllor, eller liknande för att minska hur skräpigt det är på LUDD.

\subpunkt{Framtiden för Burn-såsen}
Frida framför att blondie och wimbledon tror att året som kommer troligtvis
kommer vara det sista de kan medverka och göra sin traditionella, starka 
Burn-sås för Tacophesten.

\end{document}
