\documentclass{protokoll}

\usepackage[utf8]{inputenc}

\bild{logo.pdf}     % Ändra om man vill ha en annan bild på framsidan.
\plats{T1}          % Plats som mötet hålls i
\typ{Styrelsemöte}  %Typ av möte: Styrelsemöte|Årsmöte|medlemsmöte etc.
\datum{2018-08-29}  %Datum, form YYYY-MM-DD
\tid{18:}{18:}   %Tid: {START}{SLUT}.
\organ{Styrelsen}   %Det organ som har hand om mötet: Styrelsen|Mötet|Medlemmarna
\organisation{Luleå Universitets Datorförening}
\forordf{August Eriksson} 	            % Föreningsordförande
\ordf{August Eriksson}%[titel] 	    % Mötesordförande  | (titel)
\sekr{Jens Lindholm}%[titel] 		% Mötessekreterare | (titel)
\justA{namn}{titel} 		% Justerare A      | titel
\justB{namn}{titel} 		% Justerare B      | titel

% Lista över övriga personer som närvarade vid mötet.
% Om det anses för många vid ett möte skriv: \person{<ANTAL> övriga medlemmar.}{}
\person{Fredrik Pettersson}{Kassör}
\person{Anton Johansson}{Ledamot}
\person{Edvin Åkerström}{Ledamot}

%\adjung{} % Adjungerad person
%\adjung{} % Nästa adjungerade person
%\behorighet{Trump} % Om mötet är obehörigt, inkludera denna rad med anledning(-arna).

\begin{document}
% Punkter som hanteras automagiskt:
% Mötets högtidliga öppnande - när mötet öppnats
% Formalia
%       Val av mötesordförande - den som valdes tills mötesordförande
%       Val av mötessekreterare - den som valdes till mötessekreterare (alltså du)
%       Val avtvå justeringspersoner tillika rösträknare - de som valdes till protokolljusterare
%       Mötets behöriga utlysande - om mötet är behörigt är denna automagisk annars använd \behorighet
%       (Eventuella adjungeringar) - en samling av de som inadjungerats; kontrolleras via \adjung kommadot
% Mötets högtidliga avslutande - när mötet avslutats
\newpage  


\punkt{Godkännande av dagordning}
% Skriv eventuella ändringar i dagordningen och ändra beslutet.
\begin{beslut}
     \att fastslå dagordningen som skickats ut.
\end{beslut}

\punkt{Bordlagda ärenden}
% Har det bordlagts ärenden från förra mötet?
Inga bordlagda ärenden.

\punkt{Nolleperioden}
% Skriv det som är relevant kring punkten.

\punkt{Pågående projekt}

\subpunkt{LEDs i korridor}

\subpunkt{Profilering}

\subpunkt{Ny statisk hemsida}

\subpunkt{Övriga projekt}

\punkt{Ny boschansvarig}
% Skriv det som är relevant kring punkten.
% Skriv det som är relevant kring punkten.
\punkt{Inköp}

\subpunkt{Ny frys}

\subpunkt{Soppåsar}

\subpunkt{Färg till målning}

\subpunkt{Övrigt}



\punkt{Målning av logga utomhus}

\punkt{Övriga ärenden}
% Icke beslutsfattande frågor och funderingar kring föreningen.
Inga övriga frågor.


\end{document}
