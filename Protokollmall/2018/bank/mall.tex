\documentclass{protokoll}

\usepackage[utf8]{inputenc}

\bild{logo} % Ändra om man vill ha en annan bild på framsidan
\plats{T1}
\typ{Styrelsemöte}  %Styrelsemöte|Årsmöte|medlemsmöte etc.
\datum{2017-12-11}  %Datum, form 0000-00-00
\tid{18:08}{19:12}  %Tid, på formen 12:34. Använd search/replace.
\organ{Styrelsen}   %Styrelsen|Mötet|Medlemmarna etc.
\organisation{Luleå Universitets Datorförening}
\forordf{Gustaf Elf} 	% Föreningsordförande
\ordf{Gustaf Elf}%[titel] 	    % Mötesordförande  | (titel)
\sekr{John Elfberg Larsson}%[titel] 		% Mötessekreterare | (titel)
\justA{Erik Viklund}{Ledamot} 		% Justerare A      | titel
\justB{Jens Lindholm}{Ledamot} 		% Justerare B      | titel

\person{Emil Kitti}{Huvudsystemansvarig}
\person{Edvin Åkerfeldt}{Kassör}
\person{Johan Jatko}{Root}

%\adjung{} % Adjungerad person
%\behorighet{Trump} % Om mötet är obehörigt, inkludera denna rad med anledning(-arna).

\newcommand*{\SignatureAndDate}[1]{%
    \vspace{10pt}
    \par\noindent\makebox[8cm]{\dotfill} %\hfill\makebox[2.0in]{\dotfill}%
    \par\noindent\makebox[8cm][r]{#1}     % \hfill\makebox[2.0in][l]{Date}%
}%

\begin{document}

\newpage  
\punkt{Godkännande av dagordning}

% Skriv ändringar och ändra beslutet
\begin{beslut}
    \att~fastslå dagordningen som skickats ut med tillägget av punkten \textit{Medlemsmöte} under punkten \textit{Bankärenden}.
\end{beslut}

\punkt{Bordlagda ärenden}
% Ändra vid cid behov.
Inga bordlagda ärenden.

\punkt{Uppdateringar av pågående projekt}
% Vad har föreningen gjort med pågående projekt.
\subpunkt{Profilering}
Oktagonen har ej målats. Ledamöterna får uppgiften att måla oktagonen så snart som möjligt.

\punkt{\LaTeX-kurs}
Edvin tog upp att han var nervös, men tyckte föreläsningen gick bra.

John nämnde att labbdelen gick bra och de flesta som var där verkade göra den och uppskatta den. Han tyckte dock att labbarna var lite slarvigt gjorda och upplagda. Till nästa kurs borde en lättnavigerad hemsida göras, så nybörjare slipper navigera ett git repo.

Det var 16 st närvarande (inklusive handledare) på både föreläsning och workshoppen.

\punkt{Trappa}
Inget har gjorts på trappan ännu. Styrelsen teoriserade att efter jul kommer den troligtvis börja byggas på.

\punkt{Bankärenden}
Edvin tog upp att Swedbank kräver att de ekonomiskt ansvariga står med på ett möte där deras krav beskrivs. Edvin och Gustaf vill gärna inte ha sina personnummer fritt tillgängliga i ett protokoll.

Edvin nämnde att man kan ha ett ``censurerat'' protokoll med de fyra sista siffrorna exkluderade från det publicerade protokollet, även den fysiska kopian kommer att censureras. Detta skulle då tydligt markeras i det givna protokollet.

Edvin tog upp om bankärenden ska behandlas på årsmöten, han nämnde att Swedbank tyckte att de skulle kunna vara med på det. Styrelsen tyckte att det är lika bra att göra så snart som möjligt, men att i framtiden göra det under årsmöten.

\begin{beslut}
    \att~de ekonomiskt ansvariga, det vill säga Edvin Åkerfeldt och Gustaf Elf, är de fullmäktiga för föreningen.
\end{beslut}

Jatko tog upp att originalprotokollet kan ha personnummrena på sig oredigerade. Den censurerade versionen kan då läggas upp på hemsidan.

\begin{beslut}
    \att~ta bort eventuella personnummer från den versionen av detta protokoll som läggs upp på internet. % av redigera det protokoll som läggs upp på internet  för ordförande och kassören att ta bort de sista fyra siffrorna på versionen av protokollet som läggs upp på internet.
\end{beslut}

\subpunkt{Beslut om företrädare gentemot Swedbank AB}
\begin{beslut}
    \att~följande person/personer får:
    \begin{itemize}
        \item Företräda föreningen enligt Swedbanks \textbf{Fullmakt Ideell förening} som fullmaktshavare
        \item Skriva under blanketterna Ansökan ideell förening/Begäran om ändring för ideell förening gentemot Swedbank samt Fullmakt ideell förening, enligt ovan, som fullmaktsgivare
        \item Företräda enligt ovan \textbf{två i förening}
    \end{itemize}

    \SignatureAndDate{Namn och personnummer}
    \SignatureAndDate{Namn och personnummer}

\end{beslut}

\subpunkt{Beslut om användare av Swedbanks Internetbank}
\begin{beslut}
    \att~följande personer ska vara användare för föreningen i Swedbanks internetbank.\textit{(Behörigheter specificeras i Swedbanks blanketter för Internetbanken)}
    \SignatureAndDate{Namn och personnummer}
    \SignatureAndDate{Namn och personnummer}
\end{beslut}

\punkt{Medlemsmöte}
Det togs upp att ett medlemsmöte borde hållas så snart som möjligt för rötterna planerar att göra stora inköp.

Diskussion utbröt kring möjliga datum. Preliminärt sattes datum för medlemsmötet till den 1:a eller 8:onde februari.

\punkt{Övriga frågor}
\subpunkt{Serverhallen}
Alla LC-net servrar har flyttats och det tros vara mer balanserat i el. Downtime från torsdagkväll till söndagkväll, det ska flyttas rack för att förbereda för facebook-servrarna, som kan komma någon runt jul.

Det börjar bli slut på plats i serverhallen.

Diskussion bröt ut kring vad som kommer göras med facebook-servrarna när de väl kommer.

\end{document}
