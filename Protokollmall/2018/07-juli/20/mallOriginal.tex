\documentclass{protokoll}

\usepackage[utf8]{inputenc}

\bild{logo.pdf}     % Ändra om man vill ha en annan bild på framsidan.
\plats{T1}          % Plats som mötet hålls i
\typ{Styrelsesemöte}  %Typ av möte: Styrelsemöte|Årsmöte|medlemsmöte etc.
\datum{2018-07-20}  %Datum, form YYYY-MM-DD
\tid{18:33}{19:07}   %Tid: {START}{SLUT}.
\organ{Styrelsen}   %Det organ som har hand om mötet: Styrelsen|Mötet|Medlemmarna
\organisation{Luleå Universitets Datorförening}
\forordf{August Eriksson} 	            % Föreningsordförande
\ordf{August Eriksson}%[titel] 	    % Mötesordförande  | (titel)
\sekr{Jens Lindholm}%[titel] 		% Mötessekreterare | (titel)
\justA{Edvin Åkerström}{Ledamot} 		% Justerare A      | titel
\justB{Fredrik Pettersson}{Kassör} 		% Justerare B      | titel

% Lista över övriga personer som närvarade vid mötet.
% Om det anses för många vid ett möte skriv: \person{<ANTAL> övriga medlemmar.}{}
%\person{namn}{titel}
%\person{}{}

%\adjung{} % Adjungerad person
%\adjung{} % Nästa adjungerade person
%\behorighet{Trump} % Om mötet är obehörigt, inkludera denna rad med anledning(-arna).

\begin{document}
% Punkter som hanteras automagiskt:
% Mötets högtidliga öppnande - när mötet öppnats
% Formalia
%       Val av mötesordförande - den som valdes tills mötesordförande
%       Val av mötessekreterare - den som valdes till mötessekreterare (alltså du)
%       Val avtvå justeringspersoner tillika rösträknare - de som valdes till protokolljusterare
%       Mötets behöriga utlysande - om mötet är behörigt är denna automagisk annars använd \behorighet
%       (Eventuella adjungeringar) - en samling av de som inadjungerats; kontrolleras via \adjung kommadot
% Mötets högtidliga avslutande - när mötet avslutats
\newpage  


\punkt{Godkännande av dagordning}
% Skriv eventuella ändringar i dagordningen och ändra beslutet.
\begin{beslut}
     \att fastslå dagordningen som skickats ut.
\end{beslut}

\punkt{Uppdateringar av pågående projekt}
% Vad har föreningen gjort med pågående projekt?
\subpunkt{Trappa}
Emil Kitti, har av okänd och odokumenterad anledning, blivit befriad från sin roll som trappmästare. 
\begin{beslut}
    \att bordlägga diskussion gällande behov och ansvar kring ärendet till nästa styrelsemöte.
\end{beslut}
\subpunkt{Övriga projekt}
Inga övriga projekt.

\punkt{Bankärenden}
% Skriv det som är relevant kring punkten.
\subpunkt{Beslut om företrädare gentemot Swedbank AB}
\begin{beslut}
    \att följande personer får:
    \begin{itemize}
        \item Företräda föreningen enligt Swedbanks Fullmakt Ideell förening som fullmaktshavare
        \item Skriva under ansökan/ändringsblankett gentemot Swedbank samt fullmakt enligt ovan som fullmaktsgivare
        \item Företräda enligt ovan två i förening
    \end{itemize}
    \begin{itemize}
        \item[Namn: ] August Eriksson
        \item[Personnummer: ] 19961017--9658

        \item[Namn: ] Fredrik Pettersson
        \item[Personnummer: ] 19960919--6937
    \end{itemize}
\end{beslut}

\subpunkt{Beslut om användare av Swedbanks Internetbank}
\begin{beslut}
    \att följande personer ska vara användare för föreningen i Swedbanks internetbank. \emph{(Behörigheter specificeras i Swedbanks blanketter för Internetbanken)}
    \begin{itemize}
        \item[Namn: ] August Eriksson
        \item[Personnummer: ] 19961017--9658

        \item[Namn: ] Fredrik Pettersson
        \item[Personnummer: ] 19960919--6937
    \end{itemize}
\end{beslut}

\punkt{Övriga ärenden}
% Icke beslutsfattande frågor och funderingar kring föreningen.
\subpunkt{Rockmålning}
Fredrik Pettersson föreslår att styrelsen planerar ett datum för att samlas och göra i ordning sina styrelserockar. August Eriksson får ansvaret att planera och vara att agera tillkallande.

\subpunkt{PR-Ansvar}
Edvin Åkerström tar på sig att vara PR-ansvarig tills nästa styrelsemöte.

\subpunkt{Städning}
August Eriksson tar på sig att planera ett datum för storstädning av lokalen. Rootgruppen kommer kontaktas innan och ombes plocka undan den diverse serverutrusting som ligger spridd i lokalen.
\end{document}
